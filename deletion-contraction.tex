\chapter{Deletion and contraction}
\label{Deletion and contraction}

\section*{Definitions}


\begin{wrapfigure}[8]{r}{40 mm}
\begin{lpic}[t(-4 mm),b(0 mm),r(0 mm),l(0 mm)]{pics/deletion-contraction(1)}
\lbl[tr]{4,10;$G$}
\lbl[tr]{28,19;$G\backslash e$}
\lbl[tr]{28,5;$G/e$}
\end{lpic}
\end{wrapfigure}

Let $e$ be an edge in the pseudograph $G$.
Denote by $G\backslash e$ the pseudograph obtained from $G$ by deleting $e$,
and by $G/e$ the pseudograph obtained from $G$ by contraction the edge $e$ to a point; see the diagram.

Assume $G$ is a graph; that is, $G$ has no loops and no parallel edges.
Then so is $G\backslash e$, but $G/e$ might have parallel edges but no loops; that is, $G/e$ is a multigraph.

If $G$ is a multigraph then so is $G\backslash e$.
If the edge $e$ is parallel to $f$ in $G$, then $f$ in $G/e$ becomes a loop; that is, $G/e$ is a pseudograph in general.


\section*{Number of spanning trees}

Recall that $s(G)$ denotes the number of spanning trees in the pseudograph $G$.

{
\begin{wrapfigure}{r}{42 mm}
\begin{lpic}[t(-4 mm),b(0 mm),r(0 mm),l(0 mm)]{pics/most(1)}
\lbl[b]{21,8;bridge}
\lbl[t]{7,0;bank}
\lbl[t]{34,0;bank}
\end{lpic}
\end{wrapfigure}

\medskip

An edge $e$ in a connected graph $G$ is called \index{bridge}\emph{bridge}, if deletion of this edge makes it disconnected;
that is, the remaining graph has two connected components which are called {}\emph{banks}.


\begin{thm}{Exercise}
Assume that the graph $G$ contains a bridge between banks $H_1$ and $H_2$.
Show that
\[s(G)=s(H_1)\cdot s(H_2).\]
\end{thm}

}


\begin{thm}{Deletion-plus-contraction formula}\index{deletion-plus-contraction formula}
\label{thm:deletion-plus-contraction}
Let $e$ be an edge in the pseudograph $G$.
Assume $e$ is not a loop, then the following identity holds
\[s(G)=s(G\backslash e)+s(G/e).\eqlbl{eq:deletion-plus-contraction}\]

\end{thm}

\begin{wrapfigure}[9]{r}{40 mm}
\begin{lpic}[t(-2 mm),b(0 mm),r(0 mm),l(0 mm)]{pics/osnovnoe-ravenstvo(1)}
\lbl[tr]{4,10;$G$}
\lbl[br]{28,26;$G\backslash e$}
\lbl[tr]{28,5;$G/e$}
\end{lpic}
\end{wrapfigure}

Often it is convenient to write the identity \ref{eq:deletion-plus-contraction} using a diagram as on the picture;
the edge $e$ is marked on the diagram.

\parit{Proof.} Note that the spanning trees of $G$ can be subdivided into two groups ---
(1)~those which contain the edge $e$ and (2)~those which do not.
For the trees in the first group, contraction of $e$ to a point  gives a spanning tree in~$G/e$, while the trees in the second group are also spanning trees in~$G\backslash e$.

Moreover, both of the described correspondences are one-to-one.
Hence the formula follows.
\qeds

Note that a spanning tree can not have loops.
Therefore if we remove all loops from the pseudograph, then the number of spanning trees remains unchanged.
In other words, for any loop $e$ the following identity holds 
\[s(G)=s(G\backslash e).\]

From the deletion-plus-contraction formula we can deduce few other useful identities.
For example, assume that the graph $G$ has an end vertex $w$ (that is, $\deg w=1$). 
If we remove the vertex $w$ and its edge from $G$, then in obtained graph $G\backslash w$
the number of spanning trees remains unchanged; that is,
\[s(G)=s(G\backslash w).\eqlbl{eq:deletion-plus-contraction-w}\]


Indeed, denote by $e$ the only edge incident to $w$. 
Note that the graph $G\backslash e$ is not connected, since the vertex $w$ is isolated.
Therefore 
$s(G\backslash e)=0$.
On the other hand $G/e=G\backslash w$ therefore \ref{eq:deletion-plus-contraction} implies~\ref{eq:deletion-plus-contraction-w}.

On the diagrams, we will use two-sided arrow ``$\leftrightarrow$'' for the graphs with equal number of the spanning trees.
For example, from the discussed identities we can draw the diagram, which in particular implies the following identity:
\[s(G)=2\cdot s(H).\]

\begin{wrapfigure}{r}{59 mm}
\begin{lpic}[t(-0 mm),b(0 mm),r(0 mm),l(0 mm)]{pics/diagramma(1)}
\lbl[tr]{3,15;$G$}
\lbl[tl]{54,15;$H$}
\end{lpic}
\end{wrapfigure}

Note that the deletion-plus-contraction formula gives an algorithm to calculate  the value $s(G)$ for given pseudograph $G$.
Indeed, for any edge $e$, both graphs $G\backslash e$ and $G/e$ have smaller number of edges.
That is, the deletion-plus-contraction formula reduces the problem of finding number of the trees to simpler graphs;
applying this formula few times we can reduce the question to a collection of graphs with evident answer for each.
In the next section we will show how it works.


 
\section*{Fans and their relatives}



Recall that \index{Fibonacci numbers}\emph{Fibonacci numbers} $f_n$ are defined using the recursive identity 
$f_{n+1}=f_n+f_{n-1}$
with $f_1=f_2=1$.
The sequence of Fibonacci numbers starts as
\[1,1,2,3,5,8,13,\dots\]

The graphs of the following type are called \index{fan}\emph{fans}; 
a fan with $n+1$ vertex will be denoted by $F_n$. 

\begin{center}
\begin{lpic}[t(0 mm),b(0 mm),r(0 mm),l(-10 mm)]{pics/veera(1)}
\lbl[br]{12,15;$F_1$}
\lbl[br]{27.5,16.5;$F_2$}
\lbl[br]{43.2,17.5;$F_3$}
\lbl[br]{57.5,18.5;$F_4$}
\lbl[br]{73,20;$F_5$}
\lbl[l]{83,14;{\Large$\dots$}}
\end{lpic}
\end{center}

\begin{thm}{Theorem}\label{thm:fans}
$s(F_n)=f_{2\cdot n}$.
\end{thm}

\parit{Proof.}
Applying the deletion-plus-contraction formula, we can draw the following infinite diagram. 
In addition to the fans $F_n$ we use its variations $F_n'$, which differ from $F_n$ by an extra parallel edge.
\begin{center}
\begin{lpic}[t(0 mm),b(0 mm),r(0 mm),l(0 mm)]{pics/veera-skhema(1)}
\lbl[br]{4,49;$F_6$}
\lbl[br]{42,49;$F_5$}
\lbl[br]{76,47;$F_4$}
\lbl[tr]{25,7;$F'_5$}
\lbl[tr]{63,8;$F'_4$}
\lbl[l]{77,13;{\Large$\dots$}}
\lbl[r]{17,13;{\Large$\dots$}}
\lbl[r]{0,43;{\Large$\dots$}}
\lbl[l]{85,43;{\Large$\dots$}}
\end{lpic}
\end{center}

Set $a_n=s(F_n)$ and $a'_n=s(F'_n)$.
From the diagram we get the following two recursive relations:
\begin{align*}
a_{n+1}&=a'_n+a_n,
\\
a'_n&=a_n+a'_{n-1}.
\end{align*}
That is, in the sequence 
\[a_1,a_1',a_2,a_2',a_3\dots\]
every number starting from $a_2$ is sum of previous two.

Further note that $F_1$ has two vertexes connected by unique edge,
and  $F'_1$ has two vertexes connected by a pair of parallel edges.
Hence $a_1=1=f_2$ and $a_1'=2=f_3$ and therefore 
\[a_n=f_{2\cdot n}\]
for any $n$.\qeds

\parbf{Comments.}
We can deduce a recursive relation for $a_n$, without using~$a_n'$:
\begin{align*}
a_{n+1}&=a_n'+a_n=
\\
&=2\cdot a_n+a'_{n-1}=
\\
&=3\cdot a_n-a_{n-1}.
\end{align*}
This is a special case of the called \index{constant-recursive sequences}\emph{constant-recursive sequences}.
The general term of constant-recursive sequences can be expressed by a closed formula ---
read \cite{jordan} if you wonder how.
In our case it is
\[a_n=\tfrac1{\sqrt{5}}\cdot
\left(
(\tfrac{3+\sqrt{5}}2)^n-(\tfrac{3-\sqrt{5}}2)^n
\right).\]
Since $a_n$ is integer and $0<\tfrac1{\sqrt{5}}\cdot(\tfrac{3-\sqrt{5}}2)^n<1$ for any $n\ge 1$ a shorter formula can be written
\[a_n
=
\left\lfloor\tfrac1{\sqrt{5}}\cdot(\tfrac{3+\sqrt{5}}2)^n\right\rfloor,\]
where $\lfloor x\rfloor$ denotes floor of $x$; 
that is, $\lfloor x\rfloor$ is the maximal integer that does not exceed $x$.

\begin{thm}{Exercise}
Consider the sequence of zig-zag graphs $Z_n$ of the following type:
\begin{center}
\begin{lpic}[t(1 mm),b(0 mm),r(0 mm),l(0 mm)]{pics/a-extra(1)}
\lbl[b]{5.5,5;$Z_1$}
\lbl[b]{14.5,9;$Z_2$}
\lbl[b]{24.5,11;$Z_3$}
\lbl[b]{34,13;$Z_4$}
\lbl[b]{45.5,15;$Z_5$}
\lbl[l]{49.5,8;{\Large$\dots$}}
\end{lpic}
\end{center}
Show that $s(Z_n)=f_{2\cdot n}$ for any $n$. 
\end{thm}

\parit{Hint:} Use the induction on $n$ and/or mimic the proof of Theorem~\ref{thm:fans}.


\begin{thm}{Exercise}
Let us denote by $b_n$ the number of spanning trees in the {}\emph{$n$-step ladder} $L_n$; that is, in the graph of the following type:

\begin{center}
\begin{lpic}[t(1 mm),b(0 mm),r(0 mm),l(0 mm)]{pics/a1-4(1)}
\lbl[b]{5.5,4;$L_1$}
\lbl[b]{15.5,9;$L_2$}
\lbl[b]{25.5,14;$L_3$}
\lbl[b]{35.5,19;$L_4$}
\lbl[b]{45.5,24;$L_5$}
\lbl[l]{50.5,13;{\Large$\dots$}}
\end{lpic}
\end{center}

Apply the method we used for fans $F_n$ to show that the sequence $b_n$ satisfies the following linear recursive relation
\[b_{n+1}=4\cdot b_n-b_{n-1}.\]
\end{thm}

\begin{wrapfigure}{r}{20 mm}
\begin{lpic}[t(-4 mm),b(0 mm),r(0 mm),l(0 mm)]{pics/lestnitza-shtrih(1)}
\lbl[b]{5,23.5;$L_3'$}
\lbl[b]{15,21;$L_3''$}
\end{lpic}
\end{wrapfigure}

\parit{Hint:} To construct the recursive relation, in addition to the ladders $L_n$ you will need two of its analogs $L_n'$ and $L_n''$ shown on the diagram.

\medskip

Note that $b_1=1$ и $b_2=4$; applying the exercise 
we could calculate first numbers of the sequence $(b_n)$:
\[1,4,15,56,209,780,2911,\dots \]

The following exercise is analogous, but more complicated.

\begin{thm}{Advanced exercise}
Recall that a wheel $W_n$ is the graph of following type:
\begin{center}
\begin{lpic}[t(1 mm),b(0 mm),r(0 mm),l(0 mm)]{pics/kolesa(1)}
\lbl[br]{3,11.5;$W_1$}
\lbl[b]{16,11.5;$W_2$}
\lbl[b]{30.5,11.5;$W_3$}
\lbl[b]{46,11.5;$W_4$}
\lbl[b]{61.5,11.5;$W_5$}
\lbl[l]{74,7;{\Large$\dots$}}
\end{lpic}
\end{center}
Show that the sequence $c_n=s(W_n)$ satisfies the following recursive relation
\[c_{n+1}=4\cdot c_n-4\cdot c_{n-1}+c_{n-2}.\]

\end{thm}

Using the exercise above and applying induction one can show that 
\[c_n=f_{2\cdot n+1}+f_{2\cdot n-1}-2=l_{2\cdot  n}-2\]
for any $n$;
the numbers $l_n=f_{n+1}+f_{n-1}$ are called \index{Lucas numbers}\emph{Lucas numbers};
they pop up in combinatorics as often as Fibonacci numbers.  

\section*{Remarks}

The \emph{deletion-plus-contraction formula} together with Kirchhoff's rules were used in the solution of the so called \emph{squaring the square problem}.
The history of this problem and its solution are discussed in a book of Martin Gradner \cite[Chapter 17]{gardiner}.

The proof of recurrent relation above is given by Mohammad Hassan Shirdareh Haghighi and Khodakhast Bibak in \cite{haghighi-bibak};
this problem is also discussed in a book of  Ronald Graham, Donald Knuth and Oren Patashnik \cite{knut} which is a classical book.
