\chapter{Corrections and additions}

Here we include corrections and additions to \cite{pearls}.

\section*{Correction to 3.2.2}

The proof of Theorem 3.2.2 about decomposition of a cubic graph with a bridge into 1-factors
does not explain why ``each bank has an odd number of vertexes''.

This is true since the 1-factor containing the bridge breaks all the vertexes of each bank into pairs except the end vertex of the bridge.



\section*{Addition to 7.1}

In this section, we will show that Kruskal’s algorithm actually produces a minimum-weight spanning tree in a leading partial case.
Essentially we will solve exercises 7.1.6 and 7.1.7 in \cite{pearls}. 

\begin{thm}{Theorem}
Kruskal’s algorithm produces a unique minimum weight spanning tree for a graph whose edges are labeled with
distinct weights. 

\end{thm}

In the proof we will use the following lemma:

\begin{thm}{Lemma}\label{lem:T+e-e}
Suppose that $T$ and $T'$ are two different spanning trees of a connected
graph. 
If $e$ is an edge of $T$ that is not in $T'$, then there is an edge $e'$
in $T'$ but not in $T$ with the property that 
\[T''=T' + e - e'\]
is a spanning tree of the
graph.
\end{thm}

\parit{Proof.}
Note that $T' + e$ has a cycle $C$ with $e$ in it.
Since $T$ is a tree, it can not contain $C$.
Therefore $C$ has an edge that is not in $T$;
denote it by $e'$.

The graph $T''=T' + e-e'$ is connected since it is obtained by deleting one edge from a cycle $C$ in a connected graph $T' + e$.
Clearly, $T''$ has the same number of edges as the tree $T'$.
Therefore $T''$ is a spanning tree.
\qeds

\parit{Proof of the theorem.}
Let $T$ be a spanning tree produced by Kruskal’s algorithm.
Assume 
\[e_1,\dots,e_{k-1},e_k,\dots e_{p-1}\eqlbl{eq:e-e}\] 
are the edges of $T$ listed in order of their weights
(in this order they are added by Kruskal’s algorithm).

Suppose that $T'$ is a spanning tree that minimizes weight.
Arguing by contradiction, assume $T'\ne T$.

Let $k$ be the maximal number such that the edges  $e_1,\dots,e_{k-1}$ are in $T'$ (if $k=1$, then this list is empty). 
In other words, if we list all its edges in $T'$ in order of their weights, then 
we obtain a sequence 
\[e_1,\dots,,e_{k-1},e'_k,\dots e'_{p-1}\]
with $e'_k\ne e_k$.

Note that the edge $e_k$ is in $T$, but not in $T'$.
Indeed, $e_k$ has the minimal weight among the edges that do not produce a cycle with $e_1,\dots,,e_{k-1}$;
the remaining edges have larger weights since all the weights are different.

The edge in $T'$ provided by Lemma~\ref{lem:T+e-e} lies in $T'$, but not in $T$;
therefore it must be $e'_j$ for $j\ge k$.
In particular, the weight of $e'_j$ must be bigger than the weight of $e'_k$ which, by the Kruskal’s algorithm, is bigger than the weight of $e_k$.

It follows that 
\[\mathrm{weight}(e_k)<\mathrm{weight}(e'_j).\]
Therefore $T''=T'+e_k-e'_j$ is a spanning tree with total weight smaller than $T'$, but $T'$ has the minimal total weight --- a contradiction.
\qeds

\section*{Correction to 8.4.1}

There is an inaccuracy in the proof of Theorem 8.4.1 about stretchable planar graphs.
Namely, in the planar drawing of $G-h$, the region $R$ might be unbounded.

To fix this inaccuracy, one needs to prove a slightly stronger statement.
Namely that any planar drawing of the maximal planar graph $G$ can be stretched.
That is, given a planar drawing of $G$, there is a stretched drawing of $G$ 
and a bijection between the bounded (necessarily triangular) regions such that corresponding triangles have the same edges of $G$ as the sides.

The remaining part of the proof works with no other changes.

\section*{Extra exercises}

\begin{thm}{Exercise}
Show that any critical graph with chromatic number 3 is isomorphic to an odd cycle.
\end{thm}

