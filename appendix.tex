\chapter{Corrections and additions}

Here we include corrections and additions to \cite{hartsfield-ringel}.

\section*{Correction to 3.2.2}

The proof of Theorem 3.2.2 about decomposition of a cubic graph with a bridge into 1-factors
does not explain why ``each bank has an odd number of vertexes''.

This is true since the 1-factor containing the bridge breaks all the vertexes of each bank into pairs except the end vertex of the bridge.


\section*{Correction to 8.4.1}

There is an inaccuracy in the proof of Theorem 8.4.1 about stretchable planar graphs.
Namely, in the planar drawing of $G-h$, the region $R$ might be unbounded.

To fix this inaccuracy, one needs to prove a slightly stronger statement.
Namely that any planar drawing of the maximal planar graph $G$ can be stretched.
That is, given a planar drawing of $G$, there is a stretched drawing of $G$ 
and a bijection between the bounded (necessarily triangular) regions such that corresponding triangles have the same edges of $G$ as the sides.

The remaining part of the proof works with no other changes.

\section*{Extra exercises}

\begin{thm}{Exercise}
Show $s(K_n-e)=(n-2)\cdot n^{n-3}$ for any edge $e$ in~$K_n$. 
\end{thm}

\begin{thm}{Exercise}
Show that any critical graph with chromatic number 3 is isomorphic to an odd cycle.
\end{thm}

