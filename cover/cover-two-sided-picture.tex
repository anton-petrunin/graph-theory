\documentclass{article}
\usepackage{rotating}
\usepackage[usenames,dvipsnames]{xcolor}
\usepackage[margin=0cm]{geometry}
\usepackage[T1]{fontenc}

\setlength{\unitlength}{1in}
\setlength{\fboxsep}{0mm}
\let\tmpfbox=\fbox
\let\tmpfbox=\relax

\pagestyle{empty}

\usepackage[papersize={12.689168in,9.250in
}, spine=0.15764in, cropgap=0.125in
%,cropmarks, cropframe
]{zwpagelayout}%0.125" + 6" + 84*0.002252" + 6" + .125" = 12.689168 
\linespread{1}
\definecolor{mycolor}{HTML}{868A08}%{5d793a}%{RGB}{0,190,190}%{100, 149, 237}{RGB}{250,205,25}%
\pagecolor{mycolor}

\usepackage{letltxmacro}
\LetLtxMacro\origttfamily\ttfamily
\DeclareRobustCommand*{\ttfamily}{%
  \origttfamily
  \hyphenchar\font=`\-\relax
  \fontdimen3\font=.25em\relax
  \fontdimen4\font=.167em\relax
  \fontdimen7\font=.167em\relax
}

\makeatletter
\DeclareRobustCommand\vttfamily{%
  \not@math@alphabet\vttfamily\relax
  \fontfamily{cmvtt}% cmvtt (Computer Modern) or lmvtt (Latin Modern)
  \selectfont
}
\DeclareTextFontCommand{\textvtt}{\vttfamily}
\makeatother

\begin{document}
\begin{picture}(10, 0)(1.25,8.125)

\put(7.6,7.9){{\hbox{
\hspace{3em}\parbox{.4\textwidth}{\begin{center}
\fontsize{33}{48}\usefont{OT1}{lmtt}{b}{n}
Extra pearls in graph theory                           
                                  \end{center}
}}}}

\put(9.6,7.1){\ttfamily\Large Anton Petrunin}

\put(7,.5){\ttfamily  \textsc{second edition}
}


%\put(6.37,2){\rotatebox{-90}{\ttfamily\large \textsc{anton petrunin}}}
%\put(7.4,8.4){\rotatebox{-90}{\includegraphics{../mppics/pic-187}}}

\put(.5,6.9){\hbox{
\hspace{0em}\parbox{.45\textwidth}
{\vttfamily\large 
This is a supplement for ``Pearls in graph theory'' --- a textbook written by Nora Hartsfield and Gerhard Ringel.
We discuss the following topics:
\begin{itemize}
\item Probabilistic method
\item Deletion-contraction formulas
\item Matrix theorem
\item Graph-polynomials 
\item Generating functions
\item Minimum spanning trees
\item Marriage theorem and its relatives 
\item Toroidal graphs
\item Rado graph
\end{itemize}
}}}

\put(7.3,1){\includegraphics{../mppics/pic-6}}

\put(.5,1){\includegraphics{../mppics/pic-7}}

\put(.5,.5){\ttfamily %updated on 2020-01-01\\ 
 anton-petrunin.github.io/graph-theory/
}

\end{picture}
\end{document}
