\chapter[Marriage theorem and its relatives]{Marriage theorem\\ and its relatives}
\chaptermark{Marriage theorem}

Recall that a \index{matching}\emph{matching} in a graph is a set of edges without common vertexes. 

Let $G$ be a graph and $M$ a matching in $G$.
A path $P$ in $G$ is called \index{alternated path}\emph{$M$-alternated} if the edges in $P$ 
alternate between edges from $M$ and edges not from $M$.

If an alternated path connects two unmatched vertexes of $G$, then it is called \index{augmenting path}\emph{$M$-augmenting}.
An $M$-augmenting path $P$ can be used to improve the matching $M$;
namely by deleting all the edges of $P$ in $M$
and adding the remaining edges of $P$, we obtain a matching $M'$ with more edges.
This construction implies the following:

\begin{thm}{Observation}\label{obs:augmenting}
Assume $G$ is a graph and $M$ is a maximal matching in $G$.
Then $G$ has no $M$-augmenting paths.
\end{thm} 

\begin{wrapfigure}[5]{r}{28 mm}
\vskip-4mm
\centering
\includegraphics{mppics/pic-61}
\vskip-0mm
\end{wrapfigure}

On the diagrams we denote the edges in $M$ by solid lines and the remaining edges by dashed lines.

\begin{thm}{Exercise}
Find an  augmenting path in the graph on the diagram
and use it to construct a larger matching.
\end{thm}


Recall that the term {}\emph{bigraph} is an abbreviation for a {}\emph{bipartite graph}.
The following two exercises follow from the definitions given above; 
they are the main driving forces in the Hungarian algorithm.


\begin{thm}{Exercise}
Let  $M$ be a matching in a bigraph $G$.
Show that any $M$-augmenting path connects vertexes from the opposite parts of the bigraph.
  
\end{thm}

\begin{thm}{Exercise}\label{ex:two-paths}
Let $M$ be a maximal matching in a bigraph $G$.
Assume two unmatched vertexes $l$ and $r$ lie on the opposite parts of $G$.
Show that no pair of $M$-alternated paths starting from $l$ and $r$ can have a common vertex.
\end{thm}

\section*{Marriage theorem}

Assume that $G$ is a bigraph and $S$ is a set of its vertexes.
We say that a matching $M$ of $G$ {}\emph{covers} $S$ if any vertex in $S$ is incident to an edge in $M$.

Given a set of vertexes $W$ in a graph $G$, the set $W'$ of all vertexes  adjacent to at least one of vertexes in $W$ will be called the \index{set of neighbors}\emph{set of neighbors} of $W$.
Note that if $G$ is a bigraph and $W$ lies in the left part, then $W'$ lies in the right part. 

\begin{thm}{Marriage theorem}
Let $G$ be a bigraph with the left and right parts $L$ and $R$.
Then $G$ has a matching which covers $L$ if and only if for any subset $W\subset L$ the set $W'\subset R$ of all neighbors of $W$ contains at least as many vertexes as $W$; that is, 
\[|W'|\ge |W|.\] 

\end{thm}

\parit{Proof.}
Assume that a matching $M$ is covering $L$.
Note that for any set  $W\subset L$, the set $W'$ of its neighbors includes the vertexes matched with~$W$.
In particular,
\[|W'|\ge |W|;\]
it proves the ``only if'' part.

Consider a maximal matching $M$ of $G$.
To prove the ``if'' part, it is sufficient to show that $M$ covers $L$.
Assume the contrary; that is, there is a vertex $w$ in $L$ which is not incident to any edge in $M$.

Consider the maximal set $S$ of vertexes in $G$ which are reachable from $w$ by 
$M$-alternated paths.
Denote by $W$ and $W'$ the set of left and right vertexes in $S$ correspondingly.

Since $S$ is maximal, $W'$ is the set of neighbors of $W$. 
According to Observation \ref{obs:augmenting}, the matching $M$ provides a bijection between $W-w$ and $W'$.
In particular, 
\[|W|=|W'|+1;\] 
the latter contradicts the assumption.
\qeds

\begin{thm}{Exercise}
Assume $G$ is an $r$-regular bigraph; $r\ge 1$.
Show that 
\begin{enumerate}[(a)]
\item $G$ admits a 1-factor;
\item the edge chromatic number of $G$ is $r$; in other words, $G$ can be decomposed into $1$-factors.
\end{enumerate}

\end{thm}

\parbf{Remark.}
If $r=2^n$ for an integer $n\ge 1$, then $G$ in the exercise above has an Euler's circuit. 
Note that the total number of edges in $G$ is even, so we can delete all odd edges from the circuit.
The obtained graph $G'$ is regular with degree $2^{n-1}$.
Repeating the described procedure recursively $n$ times, 
we will end up at a 1-factor of $G$.

There is a tricky way to make this idea work for arbitrary $r$, not necessarily a power of $2$; 
it was discovered by Noga Alon [see \ncite{alon} and also \ncite{17-camels}]. 

\begin{thm}{Exercise}
Children from 25 countries, 10 kids from each, decided to stand in a rectangular formation with 25 rows of 10 children in each row.
Show that you can always choose one child from each row so that all 25 of them will be from different countries.
\end{thm}

\begin{thm}{Exercise}
The sons of the king divided the kingdom between each other into 23 parts of equal area --- one for each son.
Later a new son was born. 
The king proposed a new subdivision into 24 equal parts and gave one of the parts to the newborn son.

Show that each of 23 older sons can choose a part of land in the new subdivision which overlaps with his old part.
\end{thm}

\begin{thm}{Exercise} A table $n{\times}n$ is filled with nonnegative numbers.
Assume that the sum in each column and each row is 1.
Show that one can choose $n$ cells with positive numbers which do not share columns and rows. 
\end{thm}


\begin{thm}{Advanced exercise}
In a group of people, for some fixed $s$ and any $k$,
any $k$ girls like at least $k-s$ boys in total.
Show that then all but $s$ girls may get married to the boys they like.
\end{thm}



\section*{Vertex covers}

A set $S$ of vertexes in a graph is called a \index{vertex cover}\emph{vertex cover} if any edge is incident to at least one of the vertexes in $S$.

\begin{thm}{Theorem}
In any bigraph, the number of edges in a maximal matching equals the number of vertexes in a minimal vertex cover.
\end{thm}

On the following diagram, a maximal matching is marked by solid lines;
the remaining edges of the graph are marked by dashed lines.
The vertexes of the cover are marked in black and the remaining vertexes in white;
the unmatched vertexes are marked by a cross.

\parit{Proof.}
Fix a bigraph $G$;
denote by $L$ and $R$ its left and right parts.
Let $M$ be a matching and $S$ be a vertex cover in $G$.

By the definition of vertex cover, any edge $m$ in $M$ is incident to at least one vertex in $S$.
Therefore 
\[|S|\ge |M|.\] 
That is, the number of vertexes in any vertex cover $S$ is at least as large as the number of edges in any matching $M$.

Now assume that $M$ is a maximal matching.
Let us construct a vertex cover $S$ such that $|S|\z=|M|$.

\begin{wrapfigure}{o}{30 mm}
\vskip-4mm
\centering
\includegraphics{mppics/pic-62}
\vskip2mm
\end{wrapfigure}


Denote by $U_L$ and $U_R$ the set of left and right unmatched vertexes (these are marked by cross on the diagram).
Denote by $Q_L$ and $Q_R$ the set of vertexes in $G$ which can be reached by $M$-alternated paths starting from $U_L$ and from $U_R$ correspondingly.

Note that $Q_L$ and $Q_R$ do not overlap.
Otherwise there would be an $M$-augmenting path from $U_L$ to $U_R$.
Therefore $M$ is not maximal --- a contradiction.

Further note that if $m$ is an edge in $M$, then both of its end vertexes lie either in $Q_L$ or $Q_R$, or neither.

Let us construct the set $S$ by taking one incident vertex (left or right) of each edge $m$ in $M$ by the following rule:
\emph{if $m$ connects vertexes in $Q_L$, then include its right vertex in $S$;
otherwise include its left vertex}.
Since $S$ has exactly one vertex incident to each edge of $M$, we have
\[|S|=|M|.\]

It remains to prove that $S$ is a vertex cover;
that is, at least one vertex of any edge $e$ in $G$ is in $S$.

Note that if the left vertex of $e$ lies in $Q_L$, then $e$ is an edge on an $M$-alternated path starting from $U_L$. 
Therefore the right vertex of $e$ also lies in $Q_L$.

Therefore it is sufficient to consider only the following three cases:
\begin{itemize}
\item The edge $e$ has
its right vertex in $Q_L$ and its left vertex outside of $Q_L$.
In this case, both vertexes of $e$ lie in~$S$.
\item The edge $e$ connects vertexes in $Q_L$.
In this case, the right vertex of $e$ is in $S$.
\item The edge $e$ connects vertexes outside of $Q_L$. 
In this case, the left vertex of $e$ is in~$S$.
\qeds
\end{itemize}


\begin{thm}{Exercise}
A few squares
 on a chessboard are marked.
Show that the minimal number of ranks and files that cover all marked squares is the same as the maximal number of rooks on the marked squares that do not threaten each other.
\end{thm}

\section*{Edge cover}

A collection of edges $N$ in a graph is called an \index{edge cover}\emph{edge cover} if every vertex is incident with at least one of the edges in $N$.

\begin{wrapfigure}{r}{25 mm}
\vskip-6mm
\centering
\includegraphics{mppics/pic-63}
\vskip0mm
\end{wrapfigure}

On the diagram, two edge covers of the same graph are marked in solid lines.
The second cover is minimal --- there is no edge cover with smaller number of edges.

\begin{thm}{Exercise}\label{ex:monotree}
Show that a minimal edge cover of any graph contains no paths of length 3 and no triangle.

Conclude that each component of the subgraph formed by a minimal edge cover is a \index{star}\emph{star};
that is, it is isomorphic to $K_{1,k}$ for some $k$. 
\end{thm}


\begin{thm}{Exercise}
Let $G$ be a connected graph with $p$ vertexes and $p>1$.
Assume that a minimal edge cover $N$ of $G$ contains $n$ edges, and a maximal matching $M$ of $G$ contains $m$ edges.
Show that 
\[m+n=p.\]

\end{thm}

\parit{Hint:} Show that the subgraph formed by $N$ has exactly $m$ components and use Exercise~\ref{ex:monotree}.

\section*{Minimal cut}

Recall that a \index{directed graph}\emph{directed graph} (or briefly a \index{digraph}\emph{digraph})
is a graph, where the edges have a direction associated with them;
that is, an edge in a digraph is defined as an {}\emph{ordered} pair of vertexes.

\begin{thm}{Min-cut theorem}
Let $G$ be a digraph. 
Fix vertexes $s$ and $t$ in~$G$.
Then the maximal number of oriented paths from $s$ to $t$ which do not have common edges equals to the minimal number of edges one can remove from $G$ so that there will be no oriented path from $s$ to $t$.
\end{thm}

\parit{Proof.}
Denote by $m$ the maximal number of oriented paths from $s$ to $t$ which do not have common edges.
Denote by $n$ the minimal number of edges one can remove from $G$ to disconnect $t$ from $s$; more precisely, after removing $n$ edges from $G$, there will be no oriented path from $s$ to~$t$.

Let $P_1, \dots, P_m$ be a maximal collection of oriented paths from $s$ to $t$ which have no common edges.
Note that in order to disconnected $t$ from $s$, we have to cut at least one edge in each path $P_1, \dots, P_m$.
In particular, $n\ge m$.

Consider the new orientation on $G$ where each path $P_i$ is oriented backwards --- from $t$ to $s$.

Consider the set $S$ of the vertexes which are reachable from $s$ by oriented paths for this new orientation.

Assume $S$ contains $t$; that is, there is a path $Q$ from $s$ to $t$, which
can move along $P_i$ only backwards.

\begin{wrapfigure}{o}{30 mm}
\vskip-6mm
\centering
\includegraphics{mppics/pic-64}
\vskip0mm
\end{wrapfigure}

(Further, the path $Q$ will be used the same way as the augmenting path in the proof of the marriage theorem.
In a sequence of moves, we will improve the collection $Q,P_1,\dots,P_m$ so that there will be no overlaps.
On the diagram, a case with $m=1$ that requires two moves is shown;
$P_1$ is marked by a solid line and $Q$ is marked by a dashed line.)

Since $P_1,\dots,P_m$ is a maximal collection, $Q$ overlaps with some of the paths $P_1,\dots,P_m$.
Without loss of generality, we can assume that $Q$ first overlaps with $P_1$ --- assume it meets $P_1$ at the vertex $v$ and leaves it at the vertex $w$.
Let us modify the paths $Q$ and $P_1$ the following way:
Instead of the path $P_1$ consider the path $P_1'$ that goes along $Q$ from $s$ to $v$ and after that goes along $P_1$ to $t$.
Instead of the path $Q$, consider the trail $Q'$ which goes along $P_1$ from $s$ to $w$ and after that goes along $Q$ to $t$.

If the constructed trail $Q'$ is not a path (that is, if $Q'$ visits some vertexes several times), then we can discard some circuits from $Q'$ to obtain a genuine path, 
which we will still denote by $Q'$.

Note that the obtained collection of paths $Q', P_1',P_2\dots,P_m$ satisfies the same conditions as the original collection.
Further, since we discarded the part of $P_1$ from $w$ to $v$, the total number of edges in $Q', P_1',P_2\dots,P_m$
is smaller than in the original collection $Q, P_1,P_2\z\dots,P_m$.
Therefore, by repeating the described procedure several times, we get $m+1$ paths without overlaps --- a contradiction.

\begin{wrapfigure}{o}{40 mm}
\vskip-2mm
\centering
\includegraphics{mppics/pic-65}
\vskip0mm
\end{wrapfigure}

It follows that $S\not\ni t$.

Note that all edges which connect $S$ to the remaining vertexes of $G$ are oriented toward to $S$.
That is, every such edge which comes out of $S$ in the original orientation belongs to one of the paths $P_1,\dots,P_m$.

Moreover, for each path $P_i$ there is only one such edge.
In other words, if a path $P_i$ leaves $S$, then it can not come back. 
Otherwise $S$ could be made larger by moving backwards along $P_i$.
Therefore cutting one such edge in each path $P_1,\dots,P_m$ makes it impossible to leave $S$.
In particular, we can disconnect $t$ from $s$ by cutting $m$ edges from $G$; that is, $n\le m$.
\qeds

\parbf{Remark.}
The described process has the following physical interpretation.
Think of each path $P_1,\dots,P_m$, and $Q$ like water pipelines from $s$ to $t$.
At each overlap of $Q$ with another path $P_i$, the water in $P_i$ and $Q$ runs the opposite directions.
So we can cut the overlapping edges and reconnect the open ends of the pipes to each other while keeping the water flow from $s$ to $t$ unchanged.
As the result, we get $m+1$ pipes form $s$ to $t$ with no common edges and possibly some cycles which we can discard.
\begin{figure}[h!]%{r}{30 mm}
\vskip-0mm
\centering
\includegraphics{mppics/pic-66}
\vskip0mm
\end{figure}
An example of this procedure for two paths $P_1$ and $Q$ is shown on the diagram; as above, $P_1$ is marked by solid line and $Q$ is marked by dashed line.

\begin{wrapfigure}{r}{35 mm}
\vskip-8mm
\centering
\includegraphics{mppics/pic-67}
\end{wrapfigure}

\begin{thm}{Advanced exercise}
Assume $G$ is a bigraph.
Let us add two vertexes, $s$ and $t$, to $G$ so that $s$ is connected to each vertex in the left part of $G$, and $t$ is connected to each vertex in the right part of $G$.
Orient the graph from left to right.
Denote the obtained digraph by~$\hat G$.

Give another proof of the marriage theorem for a bigraph $G$, applying the min-cut theorem to the digraph $\hat G$. 
\end{thm}


\section*{Remarks}

The marriage theorem was proved by Philip Hall in \cite{hall};
it has many applications in all branches of mathematics.
The theorem on vertex cover was discovered by D\'enes K\H{o}nig \cite{konig} and independently by Jen\H{o} Egerv\'ary \cite{egervary}.
The theorem on min-cut was proved by Peter Elias, Amiel Feinstein, and Claude Shannon \cite{EFS}, 
and independently also by Lester Ford and Delbert Fulkerson \cite{ford-fulkerson}.

An extensive overview of the marriage theorem and its relatives is given by Alexandr Evnin in \cite{evnin}.
