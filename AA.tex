
Our next aim is to figure out how to find Kirchhoff minors $K_{G\backslash e}$ and $K_{G/e}$ of the graphs $G\backslash e$ and $G/e$ from $K=K_G$.



Note that from matrix $K_G$ one can reconstruct $A'$, and therefore the original matrix $A$ as well as the multigraph $G$.
Indeed, to get $A'$ we need to add one column and row to $K$ and fill it so that the sum in each column and row of the obtained matrix vanish;
this can be done in a unique way.

In particular, from the matrix $K$ we can find the number of the spanning trees $t(G)$.
Let us denote this number by $\delta(K)$, to emphasize that $\delta(K)=t(G)$ is a function of $K$.

Note that for the function $\delta(K)$ the following conditions hold:
\begin{enumerate}
\item\label{delata-1} If we permute in $K$ a pair of columns and the pair of rows with the same numbers then for the obtained matrix $K'$ we have 
\[\delta(K)=\delta(K').\]
Indeed, $K'$ describes the same graph $G$ with an other enumeration of vertexes.

\item \label{delata-2}
If the sum of components in the first row in $K$ is positive, then
\[\delta(K)=\delta(K^{\circ})+\delta(K^{\bullet})\]
where the matrix $K^{\circ}$ obtained from $K$ by subtracting 1 from the corner component with indexes (1,1) and $K^{\bullet}$ denotes the $(n-2)\times(n-2)$-matrix, obtained from $K$ by deleting first column and first row.
This identity follows from the deletion-plus-contraction formula $({*})$ sinsce \[K^{\circ}=K_{G\backslash e}\quad\text{and}\quad K^{\bullet}=K_{G/e}.\]

\item\label{delata-3} If the sum of components in each row of $K$ vanish, then $\delta(K)\z=0$. 
Indeed, in this case the last vertex in our graph is isolated;
therefore the graph is not connected and has no spanning trees.

\item\label{delata-4} If $K$ is the unit matrix then $\delta(K)=1$;
that is
\[
\delta\left(
\begin{matrix}
1&0&\cdots&0
\\
0&1&\ddots&\vdots
\\
\vdots&\ddots&\ddots&0
\\
0&\cdots&0&1
\end{matrix}
\right)=1.
\]
Indeed, in the corresponding graph $G$ all the vertexes labeled by $1,\dots,p-1$ connected to the vertex with label $p$ by unique edge. 
Therefore the graph $G$ is a tree and hence $t(G)=1$.
\end{enumerate}

Notice that all these properties are satisfied for the determinant of $K$, which we denote further by $\det K$.

Indeed, property \ref{delata-1} holds since permuting a pair of rows or columns in $K$ changes the sign of its determinant.
Hence making two permutations leaves the determinant unchanged.
Property \ref{delata-2} follows from the cofactor expansion of the determinant of $K$;
which is assumed to be known. 
If the sum of components in each row is zero then the rows are linearly dependent;
hence property \ref{delata-3} follows.
The property \ref{delata-4} is evident.

Since these properties completely describe $\delta(K)$, we get the following theorem.



The method we used in the proof can be considered as a generalization of method of mathematical induction.
We proved equality of two numbers $t(G)$ and $\det K$ which defined in a very different way by showing that they both satisfy a list of common properties which completely defines the value $t(G)$.











Published as: Aleksandrov, A. D. (166)

     Aleksandrov, Aleksandr Danilovich (16)
     Aleksandrov, Aleksandr Danilovič (1)
     Aleksandrov, Alexandr D. (1)
     Aleksandrow, A. D. (1)
     Alexandroff, A. (6)
     Alexandroff, A. D. (13)
     Alexandrov, A. D. (12)
     Alexandrov, Aleksandr Danilovich (1)
     Alexandrov, Alexandr (1)
     Alexandrov, Alexandr D. (4)
     Alexandrov, Alexandr Danilovich (1)
     Alexandrow, A. D. (4)
