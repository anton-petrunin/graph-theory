






Let us mention another application of exponential generating functions.

Assume $r_n$ denotes the number of rooted spanning trees in $K_n$.
(A~tree with one marked vertex is called a \index{rooted tree}\emph{rooted tree} and the marked vertex is called its {}\emph{root}).
Then it is not hard to see that the exponential generating function of $r_n$ satisfies the following identity
\[R(x)=x\cdot e^{R(x)}.
\eqlbl{eq:rooted}\]
By the Lagrange inversion theorem, formula \ref{eq:rooted} implies that $r_n=n^{n-1}$.

Since in any spanning tree of $K_n$ we have $n$ choices for the root, we have that
\[r_n=n\cdot s(K_n).\]
This way we get another proof of the Cayley formula (\ref{thm:cayley}) \[s(K_n)=n^{n-2}.\]











\begin{thm}{Exercise}
Let $A$ be an abstract rewriting system
with set of objects formed by integers $\ge 2$
such that $m\to n$ if and only if $m$ is divisible by $n$ and $m>n$.
In other words, we can exchange any integer for its factor.

\begin{enumerate}[(a)]
\item Show that $m\sim n$ for any two objects $m$ and $n$ in $A$.
\item Show that $m$ is irreducible if and only if $m$ is prime.
\end{enumerate}

\end{thm}














\section{Large girth}

Recall that \index{girth}\emph{girth} of a graph is the length of its shortest cycle.

\begin{thm}{Theorem}
Any toroidal graph with girth at least $4$ has chromatic number at most $4$.
\end{thm}

The following exercise guides you to a proof.














\begin{thm}{Lemma}\label{lem:W<G}
Suppose that $W$ is a drawing of a connected pseudograph on a torus and $V$ is a connected subgraph in $W$.
Then 
\[\Sigma_W\le \Sigma_U.\eqlbl{W<G}\]
\end{thm}

\parit{Proof.} Let $V$ be a maximal connected subgraph of $W$ that contains $U$ and such that 
\[\Sigma_V\le \Sigma_U.\]

If the inequality \ref{W<G} does not hold, then $V\ne W$.
Therefore, there is an edge $e$ that lies in $W$, but not in $V$.
Since both graphs are connected we may assume that one of the vertices of $e$ is in $V$.
We have three cases:
\begin{enumerate}[(i)]
\item The other vertex of $e$ is not in $V$.
In this case we have to add this vertex as well.
Note that in this case the added edge does not separate the region containing $e$.
Therefore,
\[\Sigma_{V+e}=\Sigma_{V}.\]
\item The other vertex $e$ is in $V$ and $e$ cuts its region in two.
In this case we leave the number of vertices unchanged and number of edges and regions is increased by 1.
Therefore, 
\[\Sigma_{V+e}=\Sigma_{V}.\] 
\item\label{p-q+r<p-q+r} The other vertex $e$ is in $V$ and $e$ leaves its region connected.
In this case we leave the number of vertices and regions unchanged and number of edges is increased by 1.
Therefore, 
\[\Sigma_{V+e}=\Sigma_{V}-1.\]
\end{enumerate}
Note that in each case, the subgraph $V$ is not maximal --- a contradiction.

Hence $V=W$ and therefore \ref{W<G} holds.
\qeds

\begin{thm}{Lemma}\label{lem:W=G}
Suppose that $W$ is a drawing of a connected pseudograph on a torus and $V$ is a connected subgraph in $W$.
Suppose all regions of $V$ are simple.
Then 
\[\Sigma_W= \Sigma_U.\eqlbl{W=G}\]
\end{thm}

The proof of these two lemmas are nearly identical.

\parit{Proof.} Let $V$ be a maximal connected subgraph of $W$ that contains $U$ and such that 
\[\Sigma_V= \Sigma_U.\]
Since all regions of $U$ are simple, Claim~\ref{clm:simple-regions} implies that all regions of $V$ are simple.

If the inequality \ref{W=G} does not hold, then $V\ne W$.
Therefore, there is an edge $e$ that lies in $W$, but not in $V$.
Since both graphs are connected we may assume that one of the vertices of $e$ is in $V$.
We have three cases:
\begin{enumerate}[(i)]
\item The other vertex of $e$ is not in $V$.
In this case we have to add this vertex as well.
Note that in this case the added edge does not separate the region containing $e$.
Therefore, 
\[\Sigma_{V+e}=\Sigma_{V}.\]
\item The other vertex $e$ is in $V$.
Since all regions of $V$ are simple, $e$ separates its region in two.
In this case we leave the number of vertices unchanged and number of edges and regions is increased by 1.
Therefore, 
\[\Sigma_{V+e}=\Sigma_{V}.\] 
\end{enumerate}
Note that in each case, the subgraph $V$ is not maximal --- a contradiction.

Hence $V=W$ and therefore \ref{W<G} holds.
\qeds






















\begin{thm}{Claim}\label{clm:simple-connected}
If a pseudograph $G$ has a drawing on torus or sphere with simple regions, then $G$ is connected. 
\end{thm}


\begin{thm}{Claim}\label{clm:simple-regions}
Let $G$ be a drawing of a connected pseudograph on a torus or sphere.
Suppose that has $G$ has a subgraph $H$ with all simple regions.
Then all regions of $G$ are simple as well.
\end{thm}

















Note that $S'$ shares with $T$ one more edge than $S$.
Repeating this construction,

This inequality together with \ref{s>=t} implies that
\[\weight(s)= \weight(t).\]

Consider the path $P$ in $T$ that connects the ends of $e'$.

Note that $P$ contains an edge $e$ that is not in $S$;
such that $S'=S-e'+e$ is a spanning tree.
Indeed, $S-e'$ has two components say $S_1$ and $S_2$;
moreover the ends on $P$ lie in the different components of $S-e'$.
Therefore, there is an edge $e$ in $P$ that connects $S_0$ to $S_1$;
that is, $S-e'+e$ is a connected spanning subgraph of $G$.

Note that $S$ and $S'$ have the same number of edges and vertices.
Since $S$ is a tree, so it $S'$.

otherwise $P+e'$ would be a cycle in $S$.

Note that  $T+e'$ contains a cycle with $e$ in it.
Removing the edge $e$ from the cycle, leaves the graph connected;
therefore $T'=T+e'-e$ is a connected spanning subgraph of $G$.

By construction $T'$ is a neighbor of $T$.
Therefore
\begin{align*}
\weight(T')&=\weight(T)+\weight(e')-\weight(e)\le
\\
&\ge\weight(T).
\end{align*}
It follows that 
\[\weight(e')\ge\weight(e)\]

Further, observe that $S'=S+e-e'$ is a spanning tree.


Evidently, $T+e'-e$ and $T$ have the same number of edges.
Whence $T+e'-e$ is tree as well as $T$.

By assumption 
\[\weight(e')\ge \weight(e).\]
Therefore, 
\begin{align*}
\weight(S')&=\weight(S)+\weight(e)-\weight(e')\le
\\
&\le\weight(S).
\end{align*}

Applying the lemma, we get that $S'=S+e-e'$ is a spanning tree.
Since the $S$ has minimal weight, we have an equality in the last inequality.
That is, $S'$ is another minimum weight spanning tree in $G$.
But $S'$ shares with $T$ an additional edge ($e$);
so $S'$ has $k+1$ common edges with $T$.
But $k$ is maximal --- a contradiction.
\qeds
\qeds

In the proof we will use the following lemma.

\begin{thm}{Lemma}
Let $T$ be a spanning tree in a graph $G$.
Suppose that the ends of a path $P$ in $T$ are connected by an edge $e'$.
Then $T'=T+e'-e$ is a spanning tree in $G$.
\end{thm}

\parit{Proof.}
Note that $T+e'$ contains a cycle formed by $P$ and $e'$.
Removing the edge $e$ from the cycle, leaves the graph connected;
therefore $T'$ is a connected spanning subgraph of $G$.

Evidently, $T'$ and $T$ have the same number of edges; whence $T'$ is tree as well as $T$.
\qeds


\parit{Proof of \ref{thm:mst-iff}; ``only-if'' part.}
Suppose that the condition does not hold for a path $P$ in $T$ and edges $e$ and $e'$.
By lemma, $T'=T+e'-e$ is a spanning tree.
Since $\weight(e')<\weight(e)$, we have that
\begin{align*}
\weight(T')&=\weight(T)+\weight(e')-\weight(e)<
\\
&<\weight(T);
\end{align*}
that is, $T$ is not a minimum weight spanning tree.

\parit{``If'' part.}
Suppose a spanning tree $T$ satisfies the condition.
Consider a minimum weight spanning tree $T'$ in $G$.
If there few minimum weight spanning trees, we can assume that $T'$ is chosen so that it has maximal number of common edges with $T$; denote this number by $k$.

Arguing by contradiction, assume $T\ne T'$.
Then there is an edge $e'$ in $T'$ that is not in $T$.
Consider the path $P$ in $T$ that connects the ends of $e'$.
Note that $P$ contains an edge $e$ that is not in $T'$;
otherwise $P+e'$ would be a cycle in $T'$. 
By assumption 
\[\weight(e')\ge \weight(e).\]
Therefore, 
\begin{align*}
\weight(T'')&=\weight(T')+\weight(e)-\weight(e')\le
\\
&\le\weight(T').
\end{align*}

Applying the lemma, we get that $T''=T'+e-e'$ is a spanning tree.
Since the $T'$ has minimal weight, we have an equality in the last inequality.
That is, $T''$ is another minimum weight spanning tree in $G$.
But $T''$ shares with $T$ an additional edge ($e$);
so $T''$ has $k+1$ common edges with $T$.
But $k$ is maximal --- a contradiction.
\qeds

\begin{thm}{Exercise}
Let $G$ be a connected pseudograph with weighted edges.
Suppose all weights are different.
\begin{enumerate}[(a)]
\item Let $e$ be an edge in $G$ that has minimal weight.
Show that there is a minimum weight spanning tree $T$ in $G$ that contains $e$.
\item Given a vertex $w$, denote by $e_w$ an edge adjacent to $w$ that has minimal weight.
Show that there is a minimum weight spanning tree $T$ in $G$ that contains $e_w$.
\item Given a set of vertices $W$, denote by $e_W$ an edge that has minimal weight among all the edges connecting a vertex in $W$ to a vertex not in $W$.
Show that there is a minimum weight spanning tree $T$ in $G$ that contains $e_w$.
\end{enumerate}

\end{thm}



































The theorem below states that the obtained tree has minimal total weight.



The following lemma is the key to the proof.

\begin{thm}{Lemma}\label{lem:kruskal}
Let $G$ be a connected pseudograph with weighted edges and $e$ is a an edge in $G$ that is not a loop and has minimal weight.
Then there is a minimum weight spanning tree $T$ in $G$ that contains $e$.
\end{thm}

\parit{Proof.}
Fix a minimum weight spanning tree $T$ in $G$.
If $T$ contains $e$, then lemma follows.

Assume $e$ does not belong to $T$.
Consider the subgraph $T+e$; it contains a cycle $C$.
Removing another edge $e'$ from $C$ produces another spanning tree $T'$.
Since $e$ has minimal weight, we have $\mathrm{weight}(e)\le \mathrm{weight}(e')$.
Therefore
\begin{align*}
 \mathrm{weight}(T')&=\mathrm{weight}(T)+\mathrm{weight}(e)-\mathrm{weight}(e')<
 \\
 &\le\mathrm{weight}(T).
\end{align*}
Since $T$ has minimal total weight, so is $T'$.
By construction, $T'$ contains~$e$.
\qeds



\parit{Proof of the theorem.}
Let $T$ be a spanning tree produced by Kruskal’s algorithm.
Assume 
\[e_1,\dots,e_{k-1},e_k,\dots e_{p-1}\eqlbl{eq:e-e}\] 
are the edges of $T$ listed in order of their weights
(in this order they are added by Kruskal’s algorithm).

Suppose that $T'$ is a spanning tree.
Let $k$ be the maximal number such that the edges  $e_1,\dots,e_{k-1}$ are in $T'$ (if $k=1$, then this list is empty). 
In other words, if we list all its edges in $T'$ in order of their weights, then 
we obtain a sequence 
\[e_1,\dots,,e_{k-1},e'_k,\dots e'_{p-1}\]
with $e'_k\ne e_k$.

Assume that $T'$ minimizes weight.
Further, if there are more than one spanning tree that minimizes weight, suppose that $T'$ has maximal value $k$ defined above.

Arguing by contradiction, assume $\mathrm{weight}(T')< \mathrm{weight}(T)$.

Consider the graph $\bar G=G/\{e_1,\dots,e_{k-1}\}$;
that is, $G$ with contracted edges $e_1,\dots,e_{k-1}$.
Note that $\bar T=T/\{e_1,\dots,e_{k-1}\}$ and $\bar T'\z=T'/\{e_1,\dots,e_{k-1}\}$ are spanning trees in $\bar G$.
Further 
\begin{align*}
\mathrm{weight}(\bar T)&+\mathrm{weight}(T)-\mathrm{weight}(e_1)-\mathrm{weight}(e_{k-1}),
\\
\mathrm{weight}(\bar T')&+\mathrm{weight}(T')-\mathrm{weight}(e_1)-\mathrm{weight}(e_{k-1}).
\end{align*}
It follows that 
$\bar T'$ is a minimal weight spanning tree in $\bar G$.

Note that $e_k$ 


Note that the edge $e_k$ is in $T$, but not in $T'$; otherwise $k$ is not maximal.

The edge in $T'$ provided by Lemma~\ref{lem:T+e-e} lies in $T'$, but not in $T$;
therefore it must be $e'_j$ for $j\ge k$.
By the Kruskal’s algorithm,
\[\mathrm{weight}(e_k)\le \mathrm{weight}(e'_k)\le \mathrm{weight}(e'_j).\]
By Lemma~\ref{lem:T+e-e}, $T''=T'+e_k-e'_j$ is a sapnning tree.

If $\mathrm{weight}(e_k)< \mathrm{weight}(e'_j)$, then $\mathrm{weight}(T'')<\mathrm{weight}(T')$ --- a contradiction.

If $\mathrm{weight}(e_k)=\mathrm{weight}(e'_j)$, then $\mathrm{weight}(T'')=\mathrm{weight}(T')$, so $T''$ minimize the weight, and shares edges $e_1,\dots,e_k$ with $T$.
The latter contradicts maximality of $k$.
\qeds

\parit{Proof of \ref{thm:kruskal}.}
Let us apply induction on the number of vertices $p$.
The base $p=1$ is trivial.

Assume the statement holds for any graph with at most $p-1$ vertices.
Choose a graph $G$ with $p$ vertices.

Let $e$ is a an edge in $G$ with minimal weight.
According to the lemma, we can choose a minimum weight spanning tree $T$ that contains $e$.

Note that $T/e$ is a spanning tree in $G/e$.
Since 
\[\mathrm{weight}(T/e)=\mathrm{weight}(T)-\mathrm{weight}(e).\]
$T$ is a minimum weight spanning tree in $G$, if and only if  
$T/e$ is a minimum weight spanning tree in $G/e$.

By the induction hypothesis, 

By induction hypothesis it can be constructed using 

Assume we start with a given pseudograph $G$ and repeat the following operation until it collapses to one vertex graph:
\begin{itemize}
\item Remove all loops from $G$. Choose the edge with minimal weight and contruct it.
\end{itemize}
Note that the set of contracted edges $e_1,\dots,e_{p-1}$ from a spanning tree in the original graph.
Moreover this construction is a reformulation of Kruskal's algorithm.
In particular if $e_1,\dots,e_{p-1}$ are listed in the order of their weights,
then in the same order they are added by Kruskal’s algorithm and in the same order they have been constructed by the above procedure.




Let us describe Kruskal’s algorithm in a slightly different way:
\begin{enumerate}[1.]
\item Contrcut in $G$ an edge with minimal weight that is not a loop and include it.
If this edge connects different trees in $F$, then add it to $F$.
\item Repeat the procedure \ref{Kruskal:main} until $F$ contains $p-1$ edge.
\end{enumerate}

Let us apply induction on the number of vertices in the graph, which will be denoted by $p$.
The base case $p=1$ is trivial.

Suppose that the statement holds for all graphs with at most $p-1$ vertices.
Fix a connected weighted graph $G$ with $p$ vertices.
Let $T$ be a spanning tree in $G$ that minimize the total weight.

According to Lemma~\ref{lem:kruskal}, the edge $e$ with minimal weight belongs to $T$. 
It follows that $T/e$ is a spanning tree in $G/e$.

Note that $G/e$ has $p-1$ vertices;
therefore by the induction hypothesis, the Kruskal's algorithm produces unique minimum weight spanning tree in $G/e$.

Note that 


Suppose that $e$ is an edge in $G$ with a minimal weight.
Note that $T$ contains $e$.
Indeed, if this is not the case then $T+e$ contains a cycle $C$.
Removing any other edge $e'$ from $C$ will produce another spanning tree $T'$.
Since $e$ is minimal, $\mathrm{weight}(e')>\mathrm{weight}(e)$ and therefore $\mathrm{weight}(T')<\mathrm{weight}(T)$.
But $T$ has a minimal weight --- a contradiction.

Since $T$ contains $e$, we have that $T/e$ is a spanning tree in $G/e$.
Evidently, if $T$ minimize its weight then so does $T/e$.
The pseudograph $G/e$ has $p-1$ and therefore $T/e$ can be obtained by applying Kruskal’s algorithm to $G/e$.
\qeds
\qeds

In the proof we will use the following lemma, which is identical to \cite[Exercises 7.1.6]{hartsfield-ringel}. 

\begin{thm}{Lemma}\label{lem:T+e-e}
Suppose that $T$ and $T'$ are two different spanning trees of a connected
graph. 
If $e$ is an edge of $T$ that is not in $T'$, then there is an edge $e'$
in $T'$ but not in $T$ with the property that 
\[T''=T' + e - e'\]
is a spanning tree of the
graph.
\end{thm}

\parit{Proof.}
Note that $T' + e$ has a cycle $C$ with $e$ in it.
Since $T$ is a tree, it cannot contain $C$.
Therefore, $C$ has an edge that is not in $T$;
denote it by $e'$.

The graph $T''=T' + e-e'$ is connected since it is obtained by deleting one edge from a cycle $C$ in a connected graph $T' + e$.
Clearly, $T''$ has the same number of edges as the tree $T'$.
Therefore, $T''$ is a spanning tree.
\qeds

\parit{Proof of the theorem.}
Let $T$ be a spanning tree produced by Kruskal’s algorithm.
Assume 
\[e_1,\dots,e_{k-1},e_k,\dots e_{p-1}\eqlbl{eq:e-e}\] 
are the edges of $T$ listed in order of their weights
(in this order they are added by Kruskal’s algorithm).

Suppose that $T'$ is a spanning tree.
Let $k$ be the maximal number such that the edges  $e_1,\dots,e_{k-1}$ are in $T'$ (if $k=1$, then this list is empty). 
In other words, if we list all its edges in $T'$ in order of their weights, then 
we obtain a sequence 
\[e_1,\dots,,e_{k-1},e'_k,\dots e'_{p-1}\]
with $e'_k\ne e_k$.

Assume that $T'$ minimizes weight.
Further, if there are more than one spanning tree that minimizes weight, suppose that $T'$ has maximal value $k$ defined above.

Assume $T$ does not minimize the total weight, so $\mathrm{weight}(T')\z< \mathrm{weight}(T)$.

Note that the edge $e_k$ is in $T$, but not in $T'$; otherwise $k$ is not maximal.

The edge in $T'$ provided by Lemma~\ref{lem:T+e-e} lies in $T'$, but not in $T$;
therefore it must be $e'_j$ for $j\ge k$.
By the Kruskal’s algorithm,
\[\mathrm{weight}(e_k)\le \mathrm{weight}(e'_k)\le \mathrm{weight}(e'_j).\]
By Lemma~\ref{lem:T+e-e}, $T''=T'+e_k-e'_j$ is a sapnning tree.

If $\mathrm{weight}(e_k)< \mathrm{weight}(e'_j)$, then $\mathrm{weight}(T'')<\mathrm{weight}(T')$ --- a contradiction.

If $\mathrm{weight}(e_k)=\mathrm{weight}(e'_j)$, then $\mathrm{weight}(T'')=\mathrm{weight}(T')$, so $T''$ minimize the weight, and shares edges $e_1,\dots,e_k$ with $T$.
The latter contradicts maximality of $k$.
\qeds































Our next aim is to figure out how to find Kirchhoff minors $K_{G- e}$ and $K_{G/e}$ of the graphs $G- e$ and $G/e$ from $K=K_G$.



Note that from matrix $K_G$ one can reconstruct $A'$, and therefore the original matrix $A$ as well as the multigraph $G$.
Indeed, to get $A'$ we need to add one column and row to $K$ and fill it so that the sum in each column and row of the obtained matrix vanish;
this can be done in a unique way.

In particular, from the matrix $K$ we can find the number of the spanning trees $t(G)$.
Let us denote this number by $\delta(K)$, to emphasize that $\delta(K)=t(G)$ is a function of $K$.

Note that for the function $\delta(K)$ the following conditions hold:
\begin{enumerate}
\item\label{delata-1} If we permute in $K$ a pair of columns and the pair of rows with the same numbers, then for the obtained matrix $K'$ we have 
\[\delta(K)=\delta(K').\]
Indeed, $K'$ describes the same graph $G$ with another enumeration of vertices.

\item \label{delata-2}
If the sum of components in the first row in $K$ is positive, then
\[\delta(K)=\delta(K^{\circ})+\delta(K^{\bullet})\]
where the matrix $K^{\circ}$ obtained from $K$ by subtracting 1 from the corner component with indexes (1,1) and $K^{\bullet}$ denotes the $(n-2)\times(n-2)$-matrix, obtained from $K$ by deleting first column and first row.
This identity follows from the deletion-plus-contraction formula $({*})$ sinsce \[K^{\circ}=K_{G- e}\quad\text{and}\quad K^{\bullet}=K_{G/e}.\]

\item\label{delata-3} If the sum of components in each row of $K$ vanish, then $\delta(K)\z=0$. 
Indeed, in this case the last vertex in our graph is isolated;
therefore the graph is not connected and has no spanning trees.

\item\label{delata-4} If $K$ is the unit matrix, then $\delta(K)=1$;
that is
\[
\delta\left(
\begin{matrix}
1&0&\cdots&0
\\
0&1&\ddots&\vdots
\\
\vdots&\ddots&\ddots&0
\\
0&\cdots&0&1
\end{matrix}
\right)=1.
\]
Indeed, in the corresponding graph $G$ all the vertices labeled by $1,\dots,p-1$ connected to the vertex with label $p$ by unique edge. 
Therefore, the graph $G$ is a tree and hence $t(G)=1$.
\end{enumerate}

Notice that all these properties are satisfied for the determinant of $K$, which we denote further by $\det K$.

Indeed, property \ref{delata-1} holds since permuting a pair of rows or columns in $K$ changes the sign of its determinant.
Hence making two permutations leaves the determinant unchanged.
Property \ref{delata-2} follows from the cofactor expansion of the determinant of $K$;
which is assumed to be known. 
If the sum of components in each row is zero, then the rows are linearly dependent;
hence property \ref{delata-3} follows.
The property \ref{delata-4} is evident.

Since these properties completely describe $\delta(K)$, we get the following theorem.



The method we used in the proof can be considered as a generalization of method of mathematical induction.
We proved equality of two numbers $t(G)$ and $\det K$ which defined in a very different way by showing that they both satisfy a list of common properties which completely defines the value $t(G)$.











Published as: Aleksandrov, A. D. (166)

     Aleksandrov, Aleksandr Danilovich (16)
     Aleksandrov, Aleksandr Danilovič (1)
     Aleksandrov, Alexandr D. (1)
     Aleksandrow, A. D. (1)
     Alexandroff, A. (6)
     Alexandroff, A. D. (13)
     Alexandrov, A. D. (12)
     Alexandrov, Aleksandr Danilovich (1)
     Alexandrov, Alexandr (1)
     Alexandrov, Alexandr D. (4)
     Alexandrov, Alexandr Danilovich (1)
     Alexandrow, A. D. (4)
