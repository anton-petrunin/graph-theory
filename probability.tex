\chapter{Ramsey numbers continued}



\section*{Lower bounds}

Recall that Ramsey number $r(m,n)$ is a least positive integer for which every blue-red coloring of edges in the complete graph $K_{r(m, n)}$ contains a blue $K_m$ or a red $K_n$.

Equivalently, for any decomposition of $K_{r(m, n)}$ into two subgraphs $G$ and $H$ either $G$ contains a copy of $K_m$ or $H$ contains a copy of $K_n$. 

Therefore, in order to show that 
\[r(m,n)\ge s+1,\] 
it is sufficient to decompose $K_s$ into two subgraphs with no isomorphic copy of $K_m$ in the first one and no isomorphic copy of $K_n$ in the second one.

\begin{wrapfigure}{r}{41mm}
\begin{lpic}[t(-4 mm),b(0 mm),r(0 mm),l(0.5 mm)]{pics/lower-r-3-3(1)}
\end{lpic}
\end{wrapfigure}

For example, the subgraphs in the decomposition of $K_5$ on the diagram has no monochromatic triangles;
the latter implies that $r(3,3)\ge 6$.
We showed already that for any decomposition of $K_6$ into two subgraphs,
one of the subgraphs has a triangle;
that is, $r(3,3)=6$.

\begin{wrapfigure}{r}{41mm}
\begin{lpic}[t(-8 mm),b(0 mm),r(0 mm),l(0 mm)]{pics/lower-r-3-4(1)}
\end{lpic}
\end{wrapfigure}

Similarly, to show that $r(3,4)\ge 9$, we need to construct a decomposition of $K_{8}$ in to two subgraphs $G$ and $H$ such that $G$ contains no triangle $K_3$ and $H$ contains no  $K_4$.
In fact any decomposition of $K_9$ into two subgraphs,
first subgraph contains a triangle or the second contains a $K_4$.
That is, $r(3,4)=9$; see \cite[p. 82--83]{pearls}.

Similarly, to show that $r(4,4)\ge 18$, we need to construct a decomposition of $K_{17}$ in to two subgraphs with no $K_4$.
(In fact, $r(4,4)=18$, but we are not going to prove it.)

The corresponding decomposition is given on the following diagram.

\begin{center}
\begin{lpic}[t(-0 mm),b(0 mm),r(0 mm),l(0 mm)]{pics/lower-r-4-4(1)}
\end{lpic}
\end{center}

The constructed decomposition is rationally symmetric; the first subgraph contains the chords of angle lengths 1, 2, 4, and 8 and the second to all the cords of angle lengths 3, 5, 6 and 7.

\begin{thm}{Exercise}
Show that 

\begin{enumerate}[(a)]
\item In the decomposition of $K_8$ above, the left graph contains no triangle and the right graph contains no $K_4$.
\item In the decomposition of $K_{17}$ above, both graph contain no $K_4$.
\end{enumerate}
\end{thm}

\parit{Hint:} In each cases, fix one vertex $v$ and draw the subgraph induced by the vertexes connected to $v$.
(If uncertain, see the definition of {}\emph{induced subgraph}.)

\medskip

For larger values $m$ and $n$ the problem of finding the exact lower bound for $r(m,n)$ is quickly becomes too hard.
Even getting a reasonable estimate is challenging.
In the next section we will show how to obtain such estimate using probability.

\section*{Probabilistic method}

The probabilistic method makes possible to prove the existence of graphs with certain properties without constructing them explicitly.
The idea is to show that if one randomly chooses a graph or its coloring from a specified class, then probability that the result is of the needed property is more than zero.
The latter implies that a graph with needed property exists.

Despite that this method of proof uses probability, the final conclusion is determined for certain, without any possible error.


\medskip

Recall that $\tbinom{n}{m}$ denotes the \index{binomial coefficient}\emph{binomial coefficient};
that is, $m$ and $n$ are integers, $n\ge 0$ and 
\[\tbinom nm=\frac{n!}{m!\cdot (n-m)!}\]
if $0\le m\le n$ and $\tbinom{n}{m}=0$ otherwise.

The number $\tbinom{n}{m}$ plays an important role in combinatorics ---
it gives the number of ways that $m$ objects can be chosen from among $n$ different objects.

\begin{thm}{Theorem}\label{thm:ramsey-lower}
Assume that the inequality 
\[\tbinom N n < 2^{{\binom n 2} - 1}\]
holds for a pair of positive integers $N$ and $n$.
Then $r(n,n)>N$.
\end{thm}

\parit{Proof.} 
We need to show that the complete graph $K_N$
admits a coloring of edges in red and blue such that it has no monochromatic subgraph isomorphic to $K_n$.

Let us color the edges randomly ---
color each edge independently with probability $\tfrac12$ in red and otherwise in blue.

Fix a set $S$ of $n$ vertexes. 
Define the variable $X(S)$ to be $1$ if every edge the vertexes in $S$ has the same color, and $0$ otherwise.
Note that the number of monochromatic $n$-subgraphs in $K_N$ is the sum of $X(S)$ over all possible $n$-vertex subsets $S$. 

Note that the expected value of $X(S)$ is simply the probability that all of the $\tbinom n 2=\tfrac{n\cdot(n-1)}{2}$
edges in $S$ are the same color. 
The probability that all the edges with the ends in $S$ are blue is ${1}/{2^{\binom n 2}}$ and with the same probability all edges are red.
Since these two possibilities exclude each other the expected value of $X(S)$ is 
${2}/{2^{\binom n 2}}.$

This holds for any $n$-vertex subset $S$ of the vertexes of $K_N$.
The total number of such subsets is $\tbinom N n$.
Therefore the expected value for the sum of $X(S)$ over all $S$ is
\[X=2\cdot \tbinom N n/2^{\binom n 2}.\]

Assume that $X<1$.
Note that at least in one coloring suppose to have at most $X$ complete monochromatic $n$-subgraphs.
Since this number has to be an integer, at least one coloring must have no monochromatic $K_n$. 

Therefore if
$\tbinom N n < 2^{\binom n 2 - 1},$
then there is a coloring $K_N$ without monochromatic $n$-subgraphs.
Hence the statement follows.
\qeds

The following corollary implies that the function $n\mapsto r(n,n)$ grows at least exponentially. 

\begin{thm}{Corollary}\label{cor:2^n/2}
$r(n, n)> \tfrac1{8}\cdot 2^{\frac{n}{2}}$
for all positive integers $n\ge 2$.
\end{thm}

\parit{Proof.}
Set $N=\lfloor\tfrac1{8}\cdot 2^{\frac{n}{2}}\rfloor$;
that is, $N$ is the largest integer $\le\tfrac1{8}\cdot 2^{\frac{n}{2}}$.

Note that 
\[2^{\binom n 2 - 1}> (2^{\frac{n-3}2})^n\ge N^n.\]
and
\[\tbinom N n=\frac{N\cdot(N-1)\cdots (N-n+1)}{n!}<  N^n.\]

Therefore  
\[\tbinom N n<2^{\binom n 2 - 1}.\]
By Theorem~\ref{thm:ramsey-lower}, we get that $r(n,n)> N$.
\qeds

\begin{thm}{Exercise}
By random coloring we will understand a coloring edges of a given graph in red and blue such that each edge is colored independently in red or blue with equal chances. 

Assume the edges of the complete graph $K_{100}$ is colored randomly. 
Find expected number of monochromatic Hamiltonian cycles in $K_{100}$.
(You may use factorial in the answer.)
\end{thm} %??? find a flow in the argument???

The answer in the exercise above is a huge number bigger than $10^{125}$.
One might think that this estimate alone is sufficient to conclude that {}\emph{most} of the colorings have a monochromatic Hamiltonian cycles --- let us show that is is not that easy.
(It is still true that probability of existence of monochromatic coloring is close to 1, but the proof requires more work and it does not follow from the estimate.)

Therefore in principle it might happen that $99.99\%$ of colorings of $K_{100}$ might have no monochromatic Hamiltonian cycles and $.01\%$ of the colorings contain all the monochromatic Hamiltonian cycles.
To keep the expected value above $10^{125}$,
this $.01\%$ of colorings should have less than $10^{130}$ of monochromatic cycles in average;
the latter does not seem impossible since the total number of Hamiltonian cycles in $K_{100}$ is $99!/2>10^{155}$.

%\begin{thm}{Exercise}
%Let $A$ and $B$ be two subsets of $\{1,\dots,n\}$;
%we say that $A$ and $B$ are essentially different if the set $(A\cup B)\backslash (A\cap B)$ contains at least $\tfrac n3$ elements.
%Consider the graph $G_n$ with $2^n$ vertexes, one for each subset in the set $\{1,\dots,n\}$ where the vertexes are adjacent if the corresponding subsets are essentially different.

%Show that $G_n$ contains a complete subgraph with ??? vertexes.  
%\end{thm}



\section*{Counting proof}

In this section we will repeat the proof of Theorem~\ref{thm:ramsey-lower} using a different language, without use of probability.
We do this to affirm that probabilistic method provides real proof, without any possible error.

In principle,  any probabilistic proof admits such translation,
but in most cases, the translation is less intuitive. 

\parit{Proof of \ref{thm:ramsey-lower}.}
The graph $K_N$ has $\tbinom{N}{2}$ edges.
Each edge can be colored in blue or red therefore the total number of different colorings is \[\Omega=2^{\binom{N}{2}}.\]

Fix a subgraph isomorphic to $K_n$ in $K_N$.
Note that this graph is red in $\Omega/2^{\binom n2}$ different colorings
and yet in $\Omega/2^{\binom n2}$ different colorings this subgraph is blue.

There are $\tbinom Nn$ different subgraphs isomorphic to $K_n$ in $K_N$.
Therefore the total number of monochromatic $K_n$'s in all the colorings 
is 
\[M=\tbinom Nn\cdot\Omega\cdot  2/2^{\binom n2}.\]

If $M<\Omega$, then by the pigeonhole principle,
there is a coloring with no monochromatic $K_n$.
Hence the result.
\qeds

%from West:???8.5.8. Theorem. (Szele [1943]) Some n-vertex tournament has at least n!/2/- 1
%Hamiltonian paths.
%Proof: Generate tournaments on [n] randomly by choosing i  jar. j  i
%with equal probability for each pair Ii, j}. Let X be the number of Hamiltonian
%paths; X is the sum of n! indicator variables for the possible Hamiltonian paths.
%Each Hamiltonian path occurs with probability 1/211-1, so E(X) = n!/2"-1. In
%some tournament, X is at least as large as the expectation.
%
%This simple bound using expectation gives almost the right answer for
%the maximum number of Hamiltonian paths in an n-vertex tournament; Alan
%[1990] proved that it is at most n!/(2 - 0(1»". When almost all instances have
%a value near the extreme, probabilistic arguments are especially effective.
%Ex: Determine tJ'le expected number of vertexes of degree k in a random n-vertex
%graph with edge probability p.

\section*{Remarks}

It is unknown whether there are constants $c>0$ and $\alpha>\sqrt{2}$ such that the inequality
\[r(n,n)\ge c\cdot \alpha^n\]
holds for any $n$.
Note that by Corollary~\ref{cor:2^n/2} the inequality holds for $c=\tfrac18$ for $\alpha=\sqrt{2}$.

Similarly, it is unknown whether there is a constants $c$ and $\alpha<4$ such that the inequality
\[r(n,n)\le c\cdot \alpha^n\]
holds for any $n$.
From Theorem 4.3.2 in \cite{pearls}, the inequality holds for $c=1$ for $\alpha=4$.

The probabilistic method was introduced by Paul Erd\H os.
It  finds applications in many areas of mathematics; not only in graph theory.

Note that probabilistic method is nonconstructive ---
often when the existence of a certain graph is probed by probabilistic method,
it is still uncontrollably hard to describe a concrete example.

More involved examples of proofs based on the probabilistic method deal with {}\emph{typical properties} of random graphs.

To describe the concept, let us consider the following {}\emph{random process} which generates graph $G_p$ with $p$ vertexes.

Fix a positive integer $p$. 
Consider a graph $G_p$ with the vertexes labeled by $1,\dots,p$,
where every edge in $G_p$ exists with probability $\tfrac12$.

Note that the described process depends only on $p$ and as a result we can get any graph on $p$ vertexes with the same probability $1/2^{\binom{p}{2}}$.

Fix a property of a graph (for example connectedness)
and let $\alpha_p$ be the probability that $G_p$ has this property.
We say that the property is \index{typical property}\emph{typical} if $\alpha_p\to 1$ as $p\to \infty$.

\begin{thm}{Exercise}
Show that random graphs are typically have diameter~2.
That is, the probability that $G_p$ is has diameter~2 converges to~1 as $p\to \infty$.
\end{thm}

\parit{Hint:} Find the probability that two given vertexes lie on the distance $>2$ from each other in $G_p$; find the average number of such pairs in $G_p$; make a conclusion.

\medskip

Note that from the exercise above, it follows that in the described random process the random graphs are {}\emph{typically connected}.

The following theorem gives a deeper illustration for probabilistic method with use of typical properties,
a proof can be found in \cite[Chapter 44]{aigner-ziegler}.

\begin{thm}{Theorem}
Given a positive integer $g$ and $k$ there is a graph $G$ with girth at least $g$ and chromatic number at least $k$.
\end{thm}
