\chapter{Ramsey numbers}

Recall that Ramsey number $r(m,n)$ is a least positive integer such that every blue-red coloring of edges in the complete graph $K_{r(m, n)}$ contains a blue $K_m$ or a red $K_n$.

Switching colors in the definition shows that $r(m,n)=r(n,m)$ for any $m$ and $n$.
Therefore we may assume that $m\le n$.

Note that $r(1,n)=1$ for any positive integer $n$.
Indeed, the one-vertex graph $K_1$ has no edges;
therefore we can say that all its edges are blue (as well as red and \emph{deep green-cyan turquoise} at the same time).

\begin{thm}{Exercise}
Show that $r(2,n)=n$ for any positive integer $n$.
\end{thm}

The following table gives the values of $r(m,n)$;
it includes all currently known values for $n\ge m\ge 3$:

\begin{table}[h!]\label{ramsey-table}
\centering{%
    \begin{tabular}{|c|*{9}{c|}}
      \hline
      \diagbox[width=.8cm, height=.8cm]{$\!\!m$}{$n\!\!$}
       & 1 & 2 & 3 & 4  & 5  & 6  & 7  & 8  & 9\\
      \hline
      1& 1 & 1 & 1 & 1  & 1  & 1  & 1  & 1  & 1\\
      \hline
      2& 1 & 2 & 3 & 4  & 5  & 6  & 7  & 8  & 9\\
      \hline 
      3& 1 & 3 & 6 & 9  & 14 & 18 & 23 & 28 & 36\\
      \hline
      4& 1 & 4 & 9 & 18 & 25 & ?  & ?  & ?  & ?\\
      \hline
    \end{tabular}
  }%
\end{table}

In order to prove that $r(4,4)=18$ we have to prove a lower bound $r(4,4)\ge 18$ and an upper bound $r(4,4)\le 18$.
The inequality $r(4,4)\z\ge 18$ means that there is a blue-red coloring of edges of $K_{17}$ that has monochromatic $K_4$.
The inequality $r(4,4)\le 18$ means that in any blue-red coloring of $K_{18}$ there is has monochromatic $K_4$.

In this chapter we discuss lower bound on $r(m,n)$ for general $m$ and $n$. 

\section*{Binomial coefficients}

In this section we review properties of binomial coefficients that will be needed further.

\index{binomial coefficient}\emph{Binomial coefficients} will be denoted by $\tbinom{n}{m}$.
They can be defined as unique numbers denotes such that the identity
\[(a+b)^n=\tbinom{n}{0}\cdot a^0\cdot b^n+\tbinom{n}{1}\cdot a^1\cdot b^{n-1}+\dots +\tbinom{n}{n}\cdot a^n\cdot b^{0}\eqlbl{eq:binom-thm}
\]
holds for any real numbers $a,b$ and integer $n\ge 0$.
This identity is called \emph{binomial expansion}; it can be used to derive some identities on binomial coefficient, for example 
\[\tbinom{n}{0}+\tbinom{n}{1}+\dots +\tbinom{n}{n}=(1+1)^n=2^n.\eqlbl{eq:binom-2n}\]

The number $\tbinom{n}{m}$ plays an important role in combinatorics ---
it gives the number of ways that $m$ objects can be chosen from $n$ different objects;
this value can be found explicitly:
\[\tbinom nm=\frac{n!}{m!\cdot (n-m)!}.\]

Note that all $\tbinom{n}{m}$ different ways to choose $m$ objects from $n$ different objects are falling into two categories: (1) those which include the last object --- there are $\tbinom{n-1}{m-1}$ of them and (2) those which do not include it --- there are $\tbinom{n-1}{m}$ of them.
It follows that 
\[\tbinom{n}{m}=\tbinom{n-1}{m-1}+\tbinom{n-1}{m}.\eqlbl{eq:binomial}\]
This identity will be used in the proof of Theorem~\ref{thm:ramsey-up}.



\section*{Upper bound}

Recall that according to Theorem 4.3.2 in \cite{pearls}, the inequality
\[r(m,n) \le r(m-1, n) + r(m, n-1)\eqlbl{eq:ramsey-inq}\]
holds for all integers $m,n\ge 2$.

In other words any value $r(m,n)$ in the table above can not exceed the sum of values in the cells above and on the left from it.
The inequality \ref{eq:ramsey-inq} might be strict; for example
\[r(3,4)=9<4+6=r(2,4)+r(3,3).\]


\begin{thm}{Theorem}\label{thm:ramsey-up}
For any positive integers $m,n$ we have that  
\[r(m,n)\le \tbinom{m+n-2}{m-1}.\]
\end{thm}


\parit{Proof.}
Set 
\[s(m,n)=\tbinom{m+n-2}{m-1}=\tfrac{(m+n-2)!}{(m-1)!\cdot(n-1)!}\] so we need to show that 
\[r(m,n)\le s(m,n).\eqlbl{eq:r<s}\]
Note that from \ref{eq:binomial}, we get the identity
\[s(m,n)=s(m-1,n)+s(m,n-1)\eqlbl{eq:binomial-s}\]
which is similar to the inequality \ref{eq:ramsey-inq}.

Further note that $s(1,n)=s(n,1)=1$ for any positive integer $n$.
Indeed, $s(1,n)=\tbinom{n-1}{0}$ and there is only one choice of $0$ objects from the given $n-1$.
Similarly $s(n,1)=\tbinom{n-1}{n-1}$ and there is only one choice of $n-1$ objects from the given $n-1$.

The above observations make it possible to calculate the values of $s(m,n)$ recursively.
The following table provides some of its values.
\begin{table}[h!]
\centering{%
    \begin{tabular}{|c|*{9}{c|}}
      \hline
      \diagbox[width=.8cm, height=.8cm]{$\!\!m$}{$n\!\!$}
       & 1 & 2 & 3 & 4  & 5  & 6  & 7  & 8  & 9\\
      \hline
      1& 1 & 1 & 1 & 1  & 1  & 1  & 1  & 1  & 1\\
      \hline
      2& 1 & 2 & 3 & 4  & 5  & 6  & 7  & 8  & 9\\
      \hline 
      3& 1 & 3 & 6 & 10 & 15 & 21 & 28 & 36 & 45\\
      \hline
      4& 1 & 4 & 10& 20 & 35 & 56 & 84  & 120  & 165\\
      \hline
    \end{tabular}
  }%
\end{table}
The inequality \ref{eq:r<s} means that any value in this table can not exceed the corresponding value in the table for $r(m,n)$ on page~\pageref{ramsey-table}. 
The latter is nearly evident from \ref{eq:ramsey-inq} and \ref{eq:binomial-s};
let us show it formally.

Since
\[r(1,n)=r(n,1)=s(1,n)=s(n,1)=1,\]
the inequality \ref{eq:r<s} holds if $m=1$ or $n=1$.

Assume the inequality \ref{eq:r<s} does not hold for some $m$ and $n$.
Choose a pair $m,n$ with minimal value $m+n$ such that \ref{eq:r<s} does not hold;
from above we have that $m,n\ge2$.
Since $m+n$ is minimal, we have that
\[r(m-1,n)\le s(m-1,n)\quad \text{and}\quad r(m,n-1)\le s(m,n-1)\]
summing these two inequalities and applying \ref{eq:ramsey-inq} together with \ref{eq:binomial-s}
we get \ref{eq:r<s} --- a contradiction.
\qeds

\begin{thm}{Corollary}\label{cor:4^n}
The inequality
\[r(n,n)\le \tfrac14\cdot 4^n\] 
holds for any positive integer $n$.
\end{thm}

\parit{Proof.}
By \ref{eq:binom-2n}, we have that 
$\tbinom{k}{m}\le2^k$.
Applying Theorem~\ref{thm:ramsey-up}, we get that
\begin{align*}
r(n,n)&\le \tbinom{2\cdot n-2}{n-1}\le
\\
&\le2^{2\cdot n-2}=
\\
&=\tfrac14\cdot 4^n.
\end{align*}
\qedsf

\section*{Lower bound}

In order to show that 
\[r(m,n)\ge s+1,\] 
it is sufficient to color edges of $K_s$ in red and blue so that it has no red $K_m$ and no blue $K_n$.
Equivalently, it is sufficient to decompose $K_s$ into two subgraphs with no isomorphic copy of $K_m$ in the first one and no isomorphic copy of $K_n$ in the second one.


\begin{wrapfigure}{r}{41mm}
\centering
\begin{lpic}[t(-4 mm),b(0 mm),r(0 mm),l(0.5 mm)]{pics/lower-r-3-3(1)}
\end{lpic}
\medskip
\begin{lpic}[t(-0 mm),b(0 mm),r(0 mm),l(0 mm)]{pics/lower-r-3-4(1)}
\end{lpic}
\end{wrapfigure}

For example, the subgraphs in the decomposition of $K_5$ on the diagram has no monochromatic triangles;
the latter implies that $r(3,3)\ge 6$.
We showed already that for any decomposition of $K_6$ into two subgraphs,
one of the subgraphs has a triangle;
that is, $r(3,3)=6$.


Similarly, to show that $r(3,4)\ge 9$, we need to construct a decomposition of $K_{8}$ in to two subgraphs $G$ and $H$ such that $G$ contains no triangle $K_3$ and $H$ contains no  $K_4$.
In fact any decomposition of $K_9$ into two subgraphs,
first subgraph contains a triangle or the second contains a $K_4$.
That is, $r(3,4)=9$; see \cite[p. 82--83]{pearls}.

Similarly, to show that $r(4,4)\ge 18$, we need to construct a decomposition of $K_{17}$ in to two subgraphs with no $K_4$.
(In fact, $r(4,4)=18$, but we are not going to prove it.)

The corresponding decomposition is given on the  diagram.
The constructed decomposition is rationally symmetric; the first subgraph contains the chords of angle lengths 1, 2, 4, and 8 and the second to all the cords of angle lengths 3, 5, 6 and 7.

\begin{figure}[h!]
\centering
\begin{lpic}[t(-0 mm),b(0 mm),r(0 mm),l(0 mm)]{pics/lower-r-4-4(1)}
\end{lpic}
\end{figure}

\begin{thm}{Exercise}
Show that 

\begin{enumerate}[(a)]
\item In the decomposition of $K_8$ above, the left graph contains no triangle and the right graph contains no $K_4$.
\item In the decomposition of $K_{17}$, both graph contain no $K_4$.
\end{enumerate}
\end{thm}

\parit{Hint:}
Arguing by contradiction assume such graph exists.
Use the symmetry of the graph to conclude that it contains a given vertex $v$.
In each cases, draw the subgraph induced by the vertexes connected to $v$.
(If uncertain, see the definition of {}\emph{induced subgraph}.)


\medskip

For larger values $m$ and $n$ the problem of finding the exact lower bound for $r(m,n)$ is quickly becomes too hard.
Even getting a reasonable estimate is challenging.
In the next section we will show how to obtain such estimate using probability.

\section*{Probabilistic method}

The probabilistic method makes possible to prove the existence of graphs with certain properties without constructing them explicitly.
The idea is to show that if one randomly chooses a graph or its coloring from a specified class, then probability that the result is of the needed property is more than zero.
The latter implies that a graph with needed property exists.

Despite that this method of proof uses probability, the final conclusion is determined for certain, without any possible error.


\medskip

\begin{thm}{Theorem}\label{thm:ramsey-lower}
Assume that the inequality 
\[\tbinom N n < 2^{{\binom n 2} - 1}\]
holds for a pair of positive integers $N$ and $n$.
Then $r(n,n)>N$.
\end{thm}

\parit{Proof.} 
We need to show that the complete graph $K_N$
admits a coloring of edges in red and blue such that it has no monochromatic subgraph isomorphic to $K_n$.

Let us color the edges randomly ---
color each edge independently with probability $\tfrac12$ in red and otherwise in blue.

Fix a set $S$ of $n$ vertexes. 
Define the variable $X(S)$ to be $1$ if every edge the vertexes in $S$ has the same color, and $0$ otherwise.
Note that the number of monochromatic $n$-subgraphs in $K_N$ is the sum of $X(S)$ over all possible $n$-vertex subsets $S$. 

Note that the expected value of $X(S)$ is simply the probability that all of the $\tbinom n 2=\tfrac{n\cdot(n-1)}{2}$
edges in $S$ are the same color. 
The probability that all the edges with the ends in $S$ are blue is ${1}/{2^{\binom n 2}}$ and with the same probability all edges are red.
Since these two possibilities exclude each other the expected value of $X(S)$ is 
${2}/{2^{\binom n 2}}.$

This holds for any $n$-vertex subset $S$ of the vertexes of $K_N$.
The total number of such subsets is $\tbinom N n$.
Therefore the expected value for the sum of $X(S)$ over all $S$ is
\[X=2\cdot \tbinom N n/2^{\binom n 2}.\]

Assume that $X<1$.
Note that at least in one coloring suppose to have at most $X$ complete monochromatic $n$-subgraphs.
Since this number has to be an integer, at least one coloring must have no monochromatic $K_n$. 

Therefore if
$\tbinom N n < 2^{\binom n 2 - 1},$
then there is a coloring $K_N$ without monochromatic $n$-subgraphs.
Hence the statement follows.
\qeds

The following corollary implies that the function $n\mapsto r(n,n)$ grows at least exponentially. 

\begin{thm}{Corollary}\label{cor:2^n/2}
$r(n, n)> \tfrac1{8}\cdot 2^{\frac{n}{2}}$.
\end{thm}

\parit{Proof.}
Set $N=\lfloor\tfrac1{8}\cdot 2^{\frac{n}{2}}\rfloor$;
that is, $N$ is the largest integer $\le\tfrac1{8}\cdot 2^{\frac{n}{2}}$.

Note that 
\[2^{\binom n 2 - 1}> (2^{\frac{n-3}2})^n\ge N^n.\]
and
\[\tbinom N n=\frac{N\cdot(N-1)\cdots (N-n+1)}{n!}<  N^n.\]

Therefore  
\[\tbinom N n<2^{\binom n 2 - 1}.\]
By Theorem~\ref{thm:ramsey-lower}, we get that $r(n,n)> N$.
\qeds

\begin{thm}{Exercise}
By random coloring we will understand a coloring edges of a given graph in red and blue such that each edge is colored independently in red or blue with equal chances. 

Assume the edges of the complete graph $K_{100}$ is colored randomly. 
Find expected number of monochromatic Hamiltonian cycles in $K_{100}$.
(You may use factorial in the answer.)
\end{thm} 

\parbf{Remark.}
The answer in the exercise above is a huge number bigger than $10^{125}$.
One might think that this estimate alone is sufficient to conclude that {}\emph{most} of the colorings have a monochromatic Hamiltonian cycles --- let us show that is not that easy.
(It is still true that probability of existence of monochromatic coloring is close to 1, but the proof requires more work and it does not follow solely from the estimate.)

The total number of colorings of $K_{100}$ is $2^{\binom{100}2}>10^{1400}$.
Therefore in principle it might happen that $99.99\%$ of the colorings have no monochromatic Hamiltonian cycles and $.01\%$ of the colorings contain all the monochromatic Hamiltonian cycles.
To keep the expected value above $10^{125}$,
this $.01\%$ of colorings should have less than $10^{130}$ of monochromatic cycles in average;
the latter does not seem impossible since the total number of Hamiltonian cycles in $K_{100}$ is $99!/2>10^{155}$.

\section*{Counting proof}

In this section we will repeat the proof of Theorem~\ref{thm:ramsey-lower} using a different language, without use of probability.
We do this to affirm that probabilistic method provides real proof, without any possible error.

In principle,  any probabilistic proof admits such translation,
but in most cases, the translation is less intuitive. 

\parit{Proof of \ref{thm:ramsey-lower}.}
The graph $K_N$ has $\tbinom{N}{2}$ edges.
Each edge can be colored in blue or red therefore the total number of different colorings is \[\Omega=2^{\binom{N}{2}}.\]

Fix a subgraph isomorphic to $K_n$ in $K_N$.
Note that this graph is red in $\Omega/2^{\binom n2}$ different colorings
and yet in $\Omega/2^{\binom n2}$ different colorings this subgraph is blue.

There are $\tbinom Nn$ different subgraphs isomorphic to $K_n$ in $K_N$.
Therefore the total number of monochromatic $K_n$'s in all the colorings 
is 
\[M=\tbinom Nn\cdot\Omega\cdot  2/2^{\binom n2}.\]

If $M<\Omega$, then by the pigeonhole principle,
there is a coloring with no monochromatic $K_n$.
Hence the result.
\qeds

\section*{Graph of $\bm{n}$-cube}

In this section we give another classical application of the probabilistic method.
It requires bit more probability theory than the lower bound on Ramsey number. 

\begin{wrapfigure}{r}{25mm}
\vskip-0mm
\centering
\includegraphics{mppics/pic-3}
\vskip-0mm
\end{wrapfigure}

Let us denote by $Q_n$ the graph of the $n$-dimensional cube;
$Q_n$ has $2^n$ vertexes, each vertex is labeled by a sequence of length $n$ from zeros and ones;
two vertexes are adjacent if their labels differ only in one digit.

Graph $Q_4$ is shown on the diagram.
Note that each vertex of $Q_n$ has degree $n$.


\begin{thm}{Exercise}
Show that diameter of $Q_n$ is $n$. 
\end{thm}


\begin{thm}{Problem}\label{prob:Qn}
Suppose $\ell(n)$ denotes maximal number of vertexes in $Q_n$ on distance more than $n/3$ from each other.
Then $\ell(n)$ grows exponentially in $n$;
moreover $\ell(n)\ge 1.05^n$. 
\end{thm}

To solve the problem one has to construct a set with at least $1.05^n$ vertexes in $Q_n$ that lie far each other.
However it is hard to construct such set explicitly.
Instead we will show that if one choose that many vertexes randomly, then with positive probability they lie far each other.
To choose a random vertex in $Q_n$ one can toss a fair coins $n$ times writing each time 1 for a head and 0 for a tail and take the vertex labeled by the obtained sequence.

The following exercise guides you to a solution of the problem above.
The same argument shows that for any coefficient $k<\tfrac12$, the maximal number of vertexes in $Q_n$ on the distance larger than $k\cdot n$ from each other grows exponentially in $n$.
According to Exercise~\ref{ex:lin-Qn}, for $k= \tfrac12$ the picture is very different.

\begin{thm}{Exercise}
Let $P_n$ denotes the probability that randomly chosen vertexes in $Q_n$ lie on the distance $\le\tfrac n3$.
\begin{enumerate}[(a)]
\item Use Claim~\ref{clm:coin} to show that 
\[P_n<0.95^n.\]

\item Assume $k$ vertexes  $v_1,\dots ,v_k$ in $Q_n$ are fixed. 
Show that probability that a random vertex $v$ lie on distance larger than $\tfrac n3$ from each of $v_i$ is at least $1-k\cdot P_n$.


\item Conclude that there are at least $1.05^n$ vertexes in $Q_n$ on the distance larger than $\tfrac n3$ from each other.
\end{enumerate}
\end{thm}


\begin{thm}{Claim}\label{clm:coin}
The probability $P_n$ to obtain less than third heads after $n$ fair tosses of a coin decays exponentially in $n$;
in fact $P_n<0.95^n$ for any $n$.
\end{thm}

In the proof we will use the following observation which is called \index{Chebyshov's inequality}\emph{Chebyshov's inequality}.

Suppose $Y$ is a nonnegative random variable and $c> 0$.
Denote by $P$ the probability of the event $Y\ge c$ and by $y$ the expected value of $Y$.
Then 
\[P\cdot c\le y\eqlbl{eq:cebyshov}.\]
Indeed, consider another random variable $\bar Y$ such that $\bar Y=c$ if $Y\ge c$ and $\bar Y=0$ otherwise;
denote by $\bar y$ its expected value.
Note that $\bar Y\le Y$ and therefore $\bar y\le y$.
The random variable $\bar Y$ takes value $c$ with probability $P$ and $0$ with probability $1-P$.
Therefore $\bar y=P\cdot c$; whence \ref{eq:cebyshov} follows.


\parit{Proof.}
Let us introduce independent random variables $X_1,\dots X_n$ that returns number of heads after each toss;
each $X_i$ takes values $0$ or $1$ with probability $\tfrac12$ each.
We need to show that probability $P_n$ of the event $X_1+\dots+X_n\le\tfrac n3$ is less than 
$0.95^n$.

Consider the random variable 
\[Y=2^{-X_1-\dots-X_n};\]
denote by $y$ its expected value.

Note that $P_n$ is the probability of the event $Y\ge 2^{-\frac n3}$,
and $Y>0$ always. 
By Chebyshov's inequality, we get that
\[P_n\cdot 2^{-\frac n3}\le y.\]

The random variable $2^{-X_i}$ takes two values $1$ and $\tfrac12=2^{-1}$ with probability $\tfrac12$ each;
the expected value of $2^{-X_i}$ has to be $\tfrac12\cdot (1+\tfrac12)=\tfrac 34$.
Note that 
\[Y=2^{-X_1}\cdots 2^{-X_n}.\]
Since $X_i$ are independent, we have that
\[y=\left(\tfrac34\right)^n.\]

It follows that 
\[P_n\le \left(\tfrac34\cdot 2^{\frac13}\right)^n< 0.95^n.\]
\qedsf

\pagebreak[1]
\begin{thm}{Advanced exercise}\label{ex:lin-Qn}
\begin{enumerate}[(a)]
\item Show that $Q_n$ contains at most $2\cdot n$ vertexes on distance at least $\tfrac n2$ from each other. 
\item Show that $Q_n$ contains at most $n+1$ vertexes on distance larger than $\tfrac n2$ from each other. 
\end{enumerate}
\end{thm}
 

\section*{Remarks}

Existence of Ramsey number $r(m,n)$ for any $m$ and $n$, is the first result in the so called \emph{Ramsey theory}. 
A typical theorem in Ramsey theory roughly states that any large objects of certain type contain very ordered piece of the given size.
We recommend a book of Matthew Katz and Jan Reimann \cite{katz-reimann} on the subject. 

Corollaries \ref{cor:4^n} and \ref{cor:2^n/2} imply that 
\[\tfrac18\cdot 2^{\frac12\cdot n}\le r(n,n)\le \tfrac14\cdot 2^{2\cdot n}.\]

It is unknown if these inequalities can be essentially improved.\footnote{This question does not look impressive from the first sight, but it is considered as one of the major problems in combinatorics \cite{gowers}.}
More precisely, it is unknown whether there are constants $c>0$ and $\alpha>\tfrac12$ such that the inequality
\[r(n,n)\ge c\cdot 2^{\alpha\cdot n}\]
holds for any $n$.
Similarly, it is unknown whether there are constants $c$ and $\alpha<2$ such that the inequality
\[r(n,n)\le c\cdot 2^{\alpha\cdot n}\]
holds for any $n$.

The probabilistic method was introduced by Paul Erd\H os.
It  finds applications in many areas of mathematics; not only in graph theory.

Note that probabilistic method is nonconstructive ---
often when the existence of a certain graph is probed by probabilistic method,
it is still uncontrollably hard to describe a concrete example.

More involved examples of proofs based on the probabilistic method deal with {}\emph{typical properties} of random graphs.

To describe the concept, let us consider the following {}\emph{random process} which generates graph $G_p$ with $p$ vertexes.

Fix a positive integer $p$. 
Consider a graph $G_p$ with the vertexes labeled by $1,\dots,p$,
where every edge in $G_p$ exists with probability $\tfrac12$.

Note that the described process depends only on $p$ and as a result we can get any graph on $p$ vertexes with the same probability $1/2^{\binom{p}{2}}$.

Fix a property of a graph (for example connectedness)
and let $\alpha_p$ be the probability that $G_p$ has this property.
We say that the property is \index{typical property}\emph{typical} if $\alpha_p\to 1$ as $p\to \infty$.

\begin{thm}{Exercise}
Show that random graphs are typically have diameter~2.
That is, the probability that $G_p$ is has diameter~2 converges to~1 as $p\to \infty$.
\end{thm}

\parit{Hint:} Find the probability that two given vertexes lie on the distance $>2$ from each other in $G_p$; find the average number of such pairs in $G_p$; make a conclusion.

\medskip

Note that from the exercise above, it follows that in the described random process the random graphs are {}\emph{typically connected}.

The following theorem gives a deeper illustration for probabilistic method with use of typical properties,
a proof can be found in \cite[Chapter 44]{aigner-ziegler}.

\begin{thm}{Theorem}
Given a positive integer $g$ and $k$ there is a graph $G$ with girth at least $g$ and chromatic number at least $k$.
\end{thm}
