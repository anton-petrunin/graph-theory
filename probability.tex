\chapter{Ramsey numbers}

In this chapter, we discuss Ramsey numbers.
The will be used as an illustration of the next chapter.

\section{Ramsey numbers}

Recall that the Ramsey number $r(m,n)$ is the least positive integer such that every blue-red coloring of edges in the complete graph $K_{r(m, n)}$ contains a blue $K_m$ or a red $K_n$.

Switching colors in the definition shows that $r(m,n)=r(n,m)$ for any $m$ and $n$.
Therefore, we may assume that $m\le n$.

Note that $r(1,n)=1$ for any positive integer $n$.
Indeed, the one-vertex graph $K_1$ has no edges;
therefore we can say that all its edges are blue (as well as \textit{red} and \textit{deep green-cyan turquoise} at the same time).

\begin{thm}{Exercise}\label{ex:r(2,n)}
Show that $r(2,n)=n$ for any positive integer $n$.
\end{thm}

The following table from \cite{radziszowski} 
includes all known values of $r(m,n)$ for $n\ge m\ge 3$:

\begin{table}[ht!]\label{ramsey-table}
\centering{%
    \begin{tabular}{|c|*{9}{c|}}
      \hline
      \diagbox[width=.8cm, height=.8cm]{$\!\!m$}{$n\!\!$}
       & 1 & 2 & 3 & 4  & 5  & 6  & 7  & 8  & 9\\
      \hline
      1& 1 & 1 & 1 & 1  & 1  & 1  & 1  & 1  & 1\\
      \hline
      2& 1 & 2 & 3 & 4  & 5  & 6  & 7  & 8  & 9\\
      \hline 
      3& 1 & 3 & 6 & 9  & 14 & 18 & 23 & 28 & 36\\
      \hline
      4& 1 & 4 & 9 & 18 & 25 & ?  & ?  & ?  & ?\\
      \hline
    \end{tabular}
  }%
\end{table}

For example, the table says that $r(4,4)=18$.
In order to show this, one has to prove two inequalities $r(4,4)\z\ge 18$ and $r(4,4)\z\le 18$.
The inequality $r(4,4)\z\ge 18$ means that there is a blue-red coloring of edges of $K_{17}$ that has no monochromatic $K_4$; this will be done in Exercise~\ref{ex:K8+K17}.
The inequality $r(4,4)\le 18$ means that in any blue-red coloring of $K_{18}$ there is a monochromatic~$K_4$.
The latter follows from $r(3,4)\le 9$ via the inequality \ref{eq:ramsey-inq} below, and
a proof of $r(3,4)\le 9$ is given in \cite[4.3]{hartsfield-ringel}.

\section{Binomial coefficients}

In this section, we review properties of binomial coefficients that will be needed further.

\index{binomial coefficient}\emph{Binomial coefficients} will be denoted by $\tbinom{n}{m}$;
they can be defined as unique numbers satisfying the identity
\[(a+b)^n=\tbinom{n}{0}\cdot a^0\cdot b^n+\tbinom{n}{1}\cdot a^1\cdot b^{n-1}+\dots +\tbinom{n}{n}\cdot a^n\cdot b^{0}\eqlbl{eq:binom-thm}
\]
for any real numbers $a,b$ and integer $n\ge 0$.
This identity is called \index{binomial expansion}\emph{binomial expansion}.
It can be used to derive some identities on binomial coefficients; for example, 
\[\tbinom{n}{0}+\tbinom{n}{1}+\dots +\tbinom{n}{n}=(1+1)^n=2^n.\eqlbl{eq:binom-2n}\]

The number $\tbinom{n}{m}$ plays an important role in combinatorics, as
it gives the number of ways that $m$ objects can be chosen from $n$ different objects.
This value can be found explicitly using the formula
\[\tbinom nm=\frac{n!}{m!\cdot (n-m)!}.\]

Note that all $\tbinom{n}{m}$ different ways to choose $m$ objects from $n$ different objects fall into two categories: (1) those which include the last object --- there are $\tbinom{n-1}{m-1}$ of them, and (2) those which do not include it --- there are $\tbinom{n-1}{m}$ of them.
It follows that 
\[\tbinom{n}{m}=\tbinom{n-1}{m-1}+\tbinom{n-1}{m}.\eqlbl{eq:binomial}\]
This identity will be used in the proof of Theorem~\ref{thm:ramsey-up}.

\section{Upper bound}

Recall that according to Theorem 4.3.2 in \cite{hartsfield-ringel}, the inequality
\[r(m,n) \le r(m-1, n) + r(m, n-1)\eqlbl{eq:ramsey-inq}\]
holds for all integers $m,n\ge 2$.

In other words, any value $r(m,n)$ in the table above cannot exceed the sum of values in the cells directly above and on the left from it.
The inequality \ref{eq:ramsey-inq} might be strict; for example,
\[r(3,4)=9<4+6=r(2,4)+r(3,3).\]


\begin{thm}{Theorem}\label{thm:ramsey-up}
For any positive integers $m,n$ we have that  
\[r(m,n)\le \tbinom{m+n-2}{m-1}.\]
\end{thm}


\parit{Proof.}
Set 
\[s(m,n)=\tbinom{m+n-2}{m-1}=\tfrac{(m+n-2)!}{(m-1)!\cdot(n-1)!},\]
so we need to prove the following inequality: 
\[r(m,n)\le s(m,n).\eqlbl{eq:r<s}\]
Note that from \ref{eq:binomial}, we get the identity
\[s(m,n)=s(m-1,n)+s(m,n-1)\eqlbl{eq:binomial-s}\]
which is similar to the inequality \ref{eq:ramsey-inq}.

Further note that $s(1,n)=s(n,1)=1$ for any positive integer $n$.
Indeed, $s(1,n)=\tbinom{n-1}{0}$, and there is only one choice of $0$ objects from the given $n-1$.
Similarly $s(n,1)=\tbinom{n-1}{n-1}$, and there is only one choice of $n-1$ objects from the given $n-1$.

The above observations make it possible to calculate the values of $s(m,n)$ recursively.
The following table provides some of its
\begin{table}[ht!]
\centering{%
    \begin{tabular}{|c|*{9}{c|}}
      \hline
      \diagbox[width=.8cm, height=.8cm]{$\!\!m$}{$n\!\!$}
       & 1 & 2 & 3 & 4  & 5  & 6  & 7  & 8  & 9\\
      \hline
      1& 1 & 1 & 1 & 1  & 1  & 1  & 1  & 1  & 1\\
      \hline
      2& 1 & 2 & 3 & 4  & 5  & 6  & 7  & 8  & 9\\
      \hline 
      3& 1 & 3 & 6 & 10 & 15 & 21 & 28 & 36 & 45\\
      \hline
      4& 1 & 4 & 10& 20 & 35 & 56 & 84  & 120  & 165\\
      \hline
    \end{tabular}
  }%
\end{table}
 values.
The inequality \ref{eq:r<s} means that any value in this table cannot exceed the corresponding value in the table for $r(m,n)$ on page~\pageref{ramsey-table}. 
The latter is nearly evident from \ref{eq:ramsey-inq} and \ref{eq:binomial-s};
let us show it formally.

Since
\[r(1,n)=r(n,1)=s(1,n)=s(n,1)=1,\]
the inequality \ref{eq:r<s} holds if $m=1$ or $n=1$.

Assume the inequality \ref{eq:r<s} does not hold for some $m$ and $n$.
Choose a \index{minimal criminal}\emph{minimal criminal} pair $(m,n)$;
that is, a pair with minimal value $m+n$ such that \ref{eq:r<s} does not hold.
From above we have that $m,n\ge2$.
Since $m+n$ is minimal, we have that
\[r(m-1,n)\le s(m-1,n)\quad \text{and}\quad r(m,n-1)\le s(m,n-1)\]
summing these two inequalities and applying \ref{eq:ramsey-inq} together with \ref{eq:binomial-s}
we get \ref{eq:r<s} --- a contradiction.
\qeds

\begin{thm}{Corollary}\label{cor:4^n}
The inequality
\[r(n,n)\le \tfrac14\cdot 4^n\] 
holds for any positive integer $n$.
\end{thm}

\parit{Proof.}
By \ref{eq:binom-2n}, we have that 
$\tbinom{k}{m}\le2^k$.
Applying Theorem~\ref{thm:ramsey-up}, we get that
\begin{align*}
r(n,n)&\le \tbinom{2\cdot n-2}{n-1}\le
\\
&\le2^{2\cdot n-2}=
\\
&=\tfrac14\cdot 4^n.
\end{align*}
\qedsf

\section{Lower bound}

In order to show that 
\[r(m,n)\ge s+1,\] 
it is sufficient to color the edges of $K_s$ in red and blue so that it has no red $K_m$ and no blue $K_n$.
Equivalently, it is sufficient to decompose $K_s$ into two subgraphs with no isomorphic copies of $K_m$ in the first one and no isomorphic copies of $K_n$ in the second one.

\begin{wrapfigure}{o}{38mm}
\centering
\begin{lpic}[t(-2 mm),b(0 mm),r(0 mm),l(0 mm)]{mppics/pic-21(1)}
\end{lpic}
\bigskip
\begin{lpic}[t(-0 mm),b(0 mm),r(0 mm),l(0 mm)]{mppics/pic-22(1)}
\end{lpic}
\end{wrapfigure}

For example, the subgraphs in the decomposition of $K_5$ in the diagram have no monochromatic triangles;
this implies that $r(3,3)\ge 6$.
We already showed that for any decomposition of $K_6$ into two subgraphs,
one of the subgraphs has a triangle;
that is, $r(3,3)=6$.

Similarly, to show that $r(3,4)\ge 9$, we need to construct a decomposition of $K_{8}$ into two subgraphs $G$ and $H$ such that $G$ contains no triangle $K_3$ and $H$ contains no  $K_4$.
In fact, in any decomposition of $K_9$ into two subgraphs,
either the first subgraph contains a triangle or the second contains a $K_4$.
That is, $r(3,4)=9$ \cite[see][p. 82--83]{hartsfield-ringel}.

Applying \ref{eq:ramsey-inq}, we get 
\[r(4,4)\le 2\cdot r(3,4)=18.\]
Therefore, to prove that $r(4,4)= 18$, we need to show that  $r(4,4)\ge 18$;
that is, we need to construct a decomposition of $K_{17}$ into two subgraphs with no $K_4$.

The required decomposition is given on the  diagram.
The constructed decomposition is rationally symmetric; the first subgraph contains the chords of angle lengths 1, 2, 4, and 8 and the second contains all the chords of angle lengths 3, 5, 6, and 7.

\begin{figure}[ht!]
\centering
\begin{lpic}[t(-0 mm),b(0 mm),r(0 mm),l(0 mm)]{mppics/pic-23}
\end{lpic}
\end{figure}

\begin{thm}{Exercise}\label{ex:K8+K17}
Show that 

\begin{enumerate}[(a)]
\item In the decomposition of $K_8$ above, the left graph contains no triangle, and the right graph contains no $K_4$.
\item In the decomposition of $K_{17}$, neither graph contains any $K_4$.
\end{enumerate}
\end{thm} 

For larger values $m$ and $n$, the problem of finding the exact lower bound for $r(m,n)$ quickly becomes too hard.
Even getting a reasonable estimate is challenging.
In the next chapter we will show how to obtain such an estimate by using probability.

\chapter{Probabilistic method}

We assume that the reader is familiar with the notion of a random variable and expected value,
at least on an intuitive level.
An introductory part in any textbook on probability should be sufficient;
see for example \cite{feller, lawler, williams}.


\section{Markov's inequality}

A \index{random variable}\emph{random variable} is a real number that depends on a random event.
We will consider random variables that take only finitely many values. 

Two random variables are called \index{independent random variables}\emph{independent} if the occurrence of one does not affect the other.

The \index{expected value}\emph{expected value} is the average of a large number of independently selected outcomes of the random variable.
The expected value of a random variable $X$ will be denoted by $\EE[X]$.
Suppose a random variable $X$ takes only values $x_1,\dots,x_n$ with probabilities $P_1,\dots,P_n$, respectively,
so $P_1+\dots+P_n=1$.
Then 
\[\EE[X]=P_1\cdot x_1+\dots+P_n\cdot x_n.
\eqlbl{eq:E(X)}
\]
For example, if $X$ is the result of rolling a die, then it takes each value $1,2,\dots,6$ with probability $\tfrac16$;
therefore 
\[\EE[X]=\tfrac16\cdot 1+\dots+\tfrac16\cdot6=3.5.\]

\begin{thm}{Claim}\label{clm:E}
For any two random variables $X$ and $Y$ and real constant $c$ we have
\[\EE[X+Y]=\EE[X]+\EE[Y]
\qquad\text{and}\qquad
\EE[c\cdot X]=c\cdot \EE[X].\eqlbl{eq:lin-E}\]
Moreover, if $X$ and $Y$ are independent, then 
\[\EE[X\cdot Y]=\EE[X]\cdot\EE[Y].\]
(We assume that all expected values in the formulas are defined.)
\end{thm}

The property in \ref{eq:lin-E} is called \emph{linearity of expectation}.
Let us emphasize that it holds \textit{without} assuming that variables are independent!
This property will be used almost everywhere in this chapter.


\begin{thm}{Markov's inequality}\index{Markov's inequality}
Suppose $Y$ is a nonnegative random variable and $c> 0$.
Denote by $P$ the probability of the event $Y\ge c$.
Then 
\[P\cdot c\le \EE[Y].
\eqlbl{eq:cebyshov}\]
\end{thm}

\parit{Proof.}
Consider another random variable $\bar Y$ such that $\bar Y=c$ if $Y\ge c$ and $\bar Y=0$ otherwise.

Note that $\bar Y\z\le Y$ and therefore
\[\EE[\bar Y]\le \EE[Y].\]
The random variable $\bar Y$ takes the value $c$ with probability $P$ and $0$ with probability $1-P$.
By \ref{eq:E(X)}, 
\[\EE[\bar Y]=P\cdot c;\] whence \ref{eq:cebyshov} follows.
\qeds


\section{Probabilistic method}

The probabilistic method allows us to prove the existence of graphs with certain properties without explicitly constructing them.
The idea is to show that if one randomly chooses a graph or its coloring from a specified class, then the probability that the result has the needed property is more than zero.
This implies that a graph with required the property exists.

Despite using probability, the final conclusion is determined for certain, without any possible error.


\medskip

\begin{thm}{Theorem}\label{thm:ramsey-lower}
Assume that the inequality 
\[\tbinom R n < 2^{{\binom n 2} - 1}\]
holds for a pair of positive integers $R$ and $n$.
Then $r(n,n)>R$.
\end{thm}

\parit{Proof.} 
We need to show that the complete graph $K_R$
admits a coloring of edges in red and blue such that it has no monochromatic subgraph isomorphic to $K_n$.

Let us color the edges randomly;
that is, color each edge independently in red or blue with equal chances.

Fix a set $S$ of $n$ vertices. 
Define the random variable $X(S)$ to be $1$ if every edge between the vertices in $S$ has the same color; otherwise set $X(S)=0$.

The random variable $X(S)$ may take the values $0$ or $1$.
The expected value of $X(S)$ is the probability that $X(S)=1$;
that is, all of the $\tbinom n 2=\tfrac{n\cdot(n-1)}{2}$
edges in $S$ have the same color. 
The probability that all the edges with the ends in $S$ are blue is ${2^{-\binom n 2}}$, and with the same probability all the edges are red.
Since these two possibilities exclude each other, 
\[\EE[X(S)]={2}\cdot {2^{-\binom n 2}}.\]
This holds for any $n$-vertex subset $S$ of the vertices of $K_R$.

Note that the number $Y$ of monochromatic $n$-subgraphs in $K_R$ is the sum of $X(S)$ over all possible $n$-vertex subsets $S$. 
The total number of such subsets is $\tbinom R n$.
By \ref{clm:E}, 
\[\EE[Y]=2\cdot \tbinom R n\cdot 2^{-\binom n 2}.\]

Denote by $P$ the probability that a random coloring of $K_R$ has at least one monochromatic $K_n$'s.
By Markov's inequality, 
\[P\le \EE[Y].\]

Since the number of monochromatic $K_n$'s is an integer, it follows that 
a random coloring of $K_R$ has no monochromatic $K_n$'s with probability at least $1-P$.
If $\EE[Y]<1$, it implies that this probability is positive;
in particular, at least one edge-coloring of $K_R$ has no monochromatic $K_n$. 
That is, if
$\tbinom R n < 2^{\binom n 2 - 1},$
then there is a coloring $K_R$ with no monochromatic $n$-subgraph.
\qeds

The following corollary implies that the function $n\mapsto r(n,n)$ grows at least exponentially. 

\begin{thm}{Corollary}\label{cor:2^n/2}
$r(n, n)> \tfrac1{8}\cdot 2^{\frac{n}{2}}$.
\end{thm}

\parit{Proof.}
Set $R=\lfloor\tfrac1{8}\cdot 2^{\frac{n}{2}}\rfloor$;
that is, $R$ is the largest integer $\le\tfrac1{8}\cdot 2^{\frac{n}{2}}$.

Note that 
\[2^{\binom n 2 - 1}> (2^{\frac{n-3}2})^n\ge R^n.\]
and
\[\tbinom R n=\frac{R\cdot(R-1)\cdots (R-n+1)}{n!}<  R^n.\]

Therefore,  
\[\tbinom R n<2^{\binom n 2 - 1}.\]
By Theorem~\ref{thm:ramsey-lower}, we get that $r(n,n)> R$.
\qeds

\begin{thm}{Exercise}\label{ex:number(ham-cycles)}
Assume the edges of the complete graph $K_{100}$ are colored randomly,
so each edge is colored independently in red or blue with equal chances. 
Show that the expected number of monochromatic Hamiltonian cycles in $K_{100}$ is larger than $10^{125}$.
\end{thm} 

\parbf{Remark.}
One might think that the exercise alone is sufficient to conclude that \textit{most} of the colorings of $K_{100}$ have a monochromatic Hamiltonian cycle.
Let us show that it is not that easy.
(It is still true that the probability of the existence of a monochromatic coloring is close to 1, but the proof requires more work.)

The total number of colorings of $K_{100}$ is $2^{\binom{100}2}>10^{1400}$.
Therefore, in principle, it might happen that $99.99\%$ of the colorings have no monochromatic Hamiltonian cycles and $0.01\%$ of the colorings contain all the monochromatic Hamiltonian cycles.
To keep the expected value above $10^{125}$,
this $0.01\%$ of the colorings should have fewer than $10^{130}$ of monochromatic cycles on average;
the latter does not seem impossible since the total number of Hamiltonian cycles in $K_{100}$ is $99!/2>10^{155}$.

\section{Counting proof}

In this section, we rewrite the proof of Theorem~\ref{thm:ramsey-lower} without the use of probability.
We do this to affirm that the probabilistic method provides a real proof, without any possible error.

In principle,  any probabilistic proof admits such a translation,
but in most cases, the translation is less intuitive. 

\parit{Counting proof of \ref{thm:ramsey-lower}.}
The graph $K_R$ has $\tbinom{R}{2}$ edges.
Each edge can be colored in blue or red;
therefore the total number of different colorings is \[\Omega=2^{\binom{R}{2}}.\]

Fix a subgraph isomorphic to $K_n$ in $K_R$.
Note that this graph is red in $\Omega/2^{\binom n2}$ different colorings
and blue in $\Omega/2^{\binom n2}$ colorings.

There are $\tbinom Rn$ different subgraphs isomorphic to $K_n$ in $K_R$.
Therefore, the total number of monochromatic $K_n$'s in all the colorings 
is 
\[M=\tbinom Rn\cdot\Omega\cdot  2/2^{\binom n2}.\]

If $M<\Omega$, then by the pigeonhole principle,
there is a coloring with no monochromatic $K_n$.
Hence the result.
\qeds

\section[\texorpdfstring{Graph of $n$-cube}{Graph of n-cube}]{Graph of $\bm{n}$-cube}

In this section, we give another classic application of the probabilistic method.

\begin{wrapfigure}{o}{25mm}
\vskip-0mm
\centering
\includegraphics{mppics/pic-24}
\vskip-0mm
\end{wrapfigure}

Let $Q_n$ denote the graph of the $n$-dimensional cube;
$Q_n$ has $2^n$ vertices, each labeled with a sequence of length $n$ consisting  of zeros and ones.
Тwo vertices are adjacent if their labels differ in only one digit.

The graph $Q_4$ is shown on the diagram.
Note that each vertex of $Q_n$ has degree $n$.

Recall that the \index{distance}\emph{distance} between two vertices in a graph $G$ is the length of a shortest path connecting the vertices.

Note that the distance between two vertices in $Q_n$ is the number of different digits in their sequences.
For example, two sequences $01101$ and $11011$ have different digits at positions $1$, $3$, and $4$,
and the corresponding vertices at distance $3$ from each other.


\begin{thm}{Problem}\label{prob:Qn}
Suppose $\ell(n)$ denotes the maximal number of vertices in $Q_n$ at a distance more than $n/3$ from each other.
Then $\ell(n)$ grows exponentially in $n$;
moreover, $\ell(n)\ge 1.05^n$. 
\end{thm}

To solve the problem, one has to construct a set with at least $1.05^n$ vertices in $Q_n$ that are far from each other.
It is challenging to construct such a set explicitly.
Instead, we will show that 
if one randomly selects that many vertices, they are far from each other with positive probability.
To choose a random vertex in $Q_n$, one can toss a coin $n$ times, writing 1 for a head and 0 for a tail, and then select the vertex labeled by the obtained sequence.

The following exercise guides you to a solution of the problem.
The same argument shows that for any coefficient $k<\tfrac12$, the maximal number of vertices in $Q_n$ at a distance larger than $k\cdot n$ from each other grows exponentially in $n$.
According to Exercise~\ref{ex:lin-Qn}, the case $k= \tfrac12$ is very different.

\begin{thm}{Exercise}\label{ex:Qn-dist}
Let $P_n$ denote the probability that two randomly chosen vertices in $Q_n$ at a distance $\le\tfrac n3$ between them.
\begin{enumerate}[(a)]

\item\label{Pn} Use Claim~\ref{clm:coin} to show that 
\[P_n<.95^n.\]

\item\label{kPn} Assume $k$ vertices  $v_1,\dots ,v_k$ in $Q_n$ are fixed and $v$ is a random vertex.
Show that $k\cdot P_n$ is the expected number of vertices $v_i$ that lies at a distance $\le\tfrac n3$ from $v$.
Use Markov's inequality to show that $v$ lies at a distance larger than $\tfrac n3$ from each of $v_i$ with probability at least $1-k\cdot P_n$.

\item\label{ex:Qn-dist:end} Apply \ref{Pn} and \ref{kPn} to show that there are at least $1.05^n$ vertices in $Q_n$ at a distance larger than $\tfrac n3$ from each other.
\end{enumerate}
\end{thm}


\begin{thm}{Claim}\label{clm:coin}
The probability $P_n$ of obtaining fewer than one-third heads after $n$ fair tosses of a coin decays exponentially in $n$;
in fact $P_n<.95^n$ for any $n$.
\end{thm}

\parit{Proof.}
Let us introduce independent $n$ random variables $X_1,\z\dots X_n$;
each $X_i$ returns the number of heads after the $i$-th toss of the coin;
in particular, each $X_i$ takes values $0$ or $1$ with the probability $\tfrac12$ each.
We need to show that the probability $P_n$ of the event $X_1+\dots+X_n\le\tfrac n3$ is less than 
$.95^n$.

Consider the random variable 
\[Y=2^{-X_1-\dots-X_n}.\]

Note that $P_n$ is the probability of the event that $Y\ge 2^{-\frac n3}$.
Also, $Y>0$. 
By Markov's inequality, we get that
\[P_n\cdot 2^{-\frac n3}\le \EE[Y].\]

The random variable $2^{-X_i}$ takes the two values $1$ and $\tfrac12=2^{-1}$ with the probability of $\tfrac12$ each.
Therefore, 
\[\EE[2^{-X_i}]=\tfrac12\cdot 1+\tfrac12\cdot\tfrac12=\tfrac 34.\]

Note that 
\[Y=2^{-X_1}\cdots 2^{-X_n}.\]
Since the random variables $X_i$ are independent, \ref{clm:E} implies that
\[\EE[Y]=\left(\tfrac34\right)^n.\]

It follows that 
\[P_n\le \left(\tfrac34\cdot 2^{\frac13}\right)^n< .95^n.\]
\qedsf

\begin{thm}{Advanced exercise}\label{ex:lin-Qn}
\begin{enumerate}[(a)]
\item \label{ex:lin-Qn:n+1} Show that $Q_n$ contains at most $n+1$ vertices at a distance larger than $\tfrac n2$ from each other.
\item \label{ex:lin-Qn:2n} Show that $Q_n$ contains at most $2\cdot n$ vertices at a distance at least $\tfrac n2$ from each other.
\end{enumerate}
\end{thm}
 

\section{Remarks}

The existence of Ramsey number $r(m,n)$ for any $m$ and $n$ is the first result in the so-called \textit{Ramsey theory}. 
A typical theorem in this theory states that any large object of a certain type contains a very ordered piece of a given size.
We recommend the book by Matthew Katz and Jan Reimann \cite{katz-reimann} on the subject. 

Corollaries \ref{cor:4^n} and \ref{cor:2^n/2} imply that 
\[\tfrac18\cdot 2^{\frac12\cdot n}\le r(n,n)\le \tfrac14\cdot 2^{2\cdot n}.\]

It is unknown if these inequalities can be essentially improved.%
\footnote{This question might look insignificant at first glance, but it is considered one of the major problems in combinatorics \cite{gowers}.}
More precisely, it is unknown whether there are constants $c>0$ and $\alpha>\tfrac12$ such that the inequality
\[r(n,n)\ge c\cdot 2^{\alpha\cdot n}\]
holds for any $n$.
Similarly, it is unknown whether there are constants $c$ and $\alpha<2$ such that the inequality
\[r(n,n)\le c\cdot 2^{\alpha\cdot n}\]
holds for any $n$.

The probabilistic method was introduced by Paul Erd\H os.
It finds applications in many areas of mathematics, not only in graph theory.

Note that the probabilistic method is nonconstructive;
if it is used to prove the existence of a certain object, then it is still uncontrollably hard to describe a concrete example.

More involved examples of proofs based on the probabilistic method deal with \textit{typical properties} of random graphs.

To describe the concept, let us consider the following \textit{random process} that generates a graph $G_n$ with $n$ vertices.

Fix a positive integer $n$. 
Consider a graph $G_n$ with the vertices labeled by $1,\dots,n$,
where the existence of an edge between every pair of vertices is decided independently by flipping a coin.

Note that the described process depends only on $n$, and as a result, we can get a graph isomorphic to any given graph with $n$ vertices.

\begin{thm}{Exercise}\label{ex:prob(isom)}
Let $H$ be a graph with $n$ vertices.
Denote by $\alpha$ the probability that $G_n$ is isomorphic to $H$.
Show that 
\[1/2^{\binom n2}\le \alpha\le n!/2^{\binom n2}.\]

Suppose $n=10$;
describe graphs $H_1$ and $H_2$ such that these inequalities become equalities.
\end{thm}


Fix a property of a graph (for example, connectedness)
and denote by $\alpha_n$ the probability that $G_n$ has this property.
We say that the property is \index{typical property}\emph{typical} if $\alpha_n\to 1$ as $n\to \infty$.

\begin{thm}{Exercise}\label{ex:diam=2}
Show that random graphs typically have a diameter of 2.
That is, the probability that $G_n$ has a diameter of 2 converges to~1 as $n\to \infty$.
\end{thm}

Note that from the exercise above, it follows that in the described random process, \textit{the random graphs are typically connected}.

\begin{thm}{Exercise}\label{ex:typ(K100)}
Show that random graphs typically have a subgraph isomorphic to $K_{100}$.
That is, the probability that $G_n$ has a subgraph isomorphic to $K_{100}$ converges to~1 as $n\to \infty$.
\end{thm}

The following theorem gives a deeper illustration of the probabilistic method with the use of typical properties;
a proof can be found in \cite[Chapter 44]{aigner-ziegler}.

\begin{thm}{Theorem}
Given positive integers $g$ and $k$, there is a graph $G$ with girth at least $g$ and a chromatic number at least $k$. %??? defs girth and chromatic number
\end{thm}
