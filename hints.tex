\backmatter

\chapter{Hints}


\raggedcolumns\setlength{\multicolsep}{10mm}
\spell{\begin{multicols}{2}}{}

\refstepcounter{chapter}
\setcounter{eqtn}{0}

\parbf{\ref{ex:cannibals}.}
Here is a part of the needed graph.

\begin{center}
\begin{tikzpicture}[scale=1.8,
  thick,main node/.style={circle,draw,font=\sffamily\bfseries,minimum size=1mm}]
  \node[main node] (1) at (0,0) {{\small${4^*_4}{\parallel}{0_0}$}};
  \node[main node] (2) at (1,-1){{\small${2_4}{\parallel}{2^*_0}$}};
  \node[main node] (3) at (0,-1){{\small${3_3}{\parallel}{1^*_1}$}};
  \node[main node] (4) at (1,-2){{\small${3^*_4}{\parallel}{1_0}$}};
    \node[main node] (5) at (-1,-1){{\small${3_4}{\parallel}{1^*_0}$}};
  \path[every node/.style={font=\sffamily\small}]
   (1) edge node[auto]{$2_0$}(2)
   (3) edge node[auto]{$1_1$}(1)
   (5) edge node[auto]{$1_0$}(1)
   (4) edge node[auto]{$0_1$}(3)
   (2) edge node[auto]{$1_0$}(4);
\end{tikzpicture}
\end{center}

\refstepcounter{chapter}
\setcounter{eqtn}{0}

\parbf{\ref{ex:r(2,n)}.}
Use the following observation:
\textit{if there is no blue $K_2$, then all edges are red.}

\parbf{\ref{ex:K8+K17}.} 
Assuming such a subgraph exists.
Fix a vertex $v$.
Note that we can assume that the subgraph contains $v$; otherwise rotate it.
In each case, draw the subgraph induced by the vertices connected to $v$.
(If uncertain, see definition of \textit{induced subgraph}.)

\refstepcounter{chapter}
\setcounter{eqtn}{0}

\parbf{\ref{ex:number(ham-cycles)}.}
Show that $K_{100}$ has $\tfrac{99!}2$ Hamiltonian cycles.
Find the probability $P$ that that a given Hamiltonian cycle is monochromatic.
Show that $N\cdot P$ is the expected number of Hamiltonian cycles.
Estimate $N\cdot P$; you can use
Stirling's inequality
\[n!>\sqrt{2\cdot \pi\cdot n}\cdot \left(\frac{n}{e}\right)^n.\]

\parbf{\ref{ex:Qn-dist};} \textit{\ref{Pn}}.
Use that the distance between two vertices is the number of different digits in their sequences.

\parit{\ref{kPn}}. Use that expected value of sum is the sum of expected values. 

\parit{\ref{ex:Qn-dist:end}}. Show and use that $1.05\cdot0.95<1$.

\parbf{\ref{ex:lin-Qn};} \textit{\ref{ex:lin-Qn:n+1}}.
Identify the vertices of $Q_n$ by $n$-dimensional vectors with $\pm1$ components.
Observe that two vertices $v$ and $w$ are at a distance $>\tfrac n2$ if and only if $\langle v,w\rangle <0$;
here $\langle\ ,\ \rangle$ denotes the scalar product (also known as dot product).

Suppose $v_1,\dots,v_k$ are $n$-dimensional vectors such that $\langle v_i,v_j\rangle <0$ if $i< j$.
Denote by $w_i$ the projection of $v_i$ to the hyperplane perpendicular to $v_k$.
Show that $\langle w_i,w_j\rangle <0$ if $i<j<k$.
Apply it with induction by $k$.

\parit{\ref{ex:lin-Qn:2n}}
Choose a hyperplane $H$ thru the center of the cube that does not pass thru any vertices of the cube.
It subdivides the vertices in our set into two collections.

Project one of these collection to $H$ and argue as in \textit{\ref{ex:lin-Qn:n+1}} to show that it has at most $n$ vertices.

\parit{Remark.}
The equality in \textit{\ref{ex:lin-Qn:2n}} holds only if $n$ is a power of 2.
Try to prove it.

\parbf{\ref{ex:prob(isom)}.} Identify vertices of $H$ and $G_n$.
Calculate probability that $H=G_n$; it gives the first inequality.
Note that there are $n!$ ways to identify vertices of $H$ and $G_n$;
it implies the second inequality.

The first inequality becomes equality if permuting vertices of $H$ gives the same graph.
There only two graphs of that type; try to find both.

The second inequality becomes equality if $H$ has no symmetries;
that is, there is no nontrivial permutation of vertices that induces an isomorphism from $H$ to itself.
Try to construct such a graph with 10 vertices.

\parbf{\ref{ex:diam=2}.}
Let $P_n$ be the probability that two vertices $v$ and $w$ in $G_n$ cannot be connected by a path of length $2$.

Find $P_n$.
Show that $\binom n2\cdot P_n$ is the expected number of such pairs.
Apply Markov's inequality.
You should get that a typical graph has diameter at most 2.

Finally, find probability that $G_n$ has diameter $<2$.
Make a conclusion.

\parbf{\ref{ex:typ(K100)}.}
Denote by $P$ the probability that the subgraph induced by $100$ vertices in $G_n$ is $K_{100}$;
note that $P>0$.

Choose disjoint 100-element subsets $S_1,\dots,S_k$ of vertices in $G_n$.
We may assume that $k>\tfrac n{100}-1$; in particular, $k\to \infty$ as $n\to \infty$.

Show that probability that at least one of $S_i$ induces a subgraph isomorphic to $K_{100}$ is at least 
\[1-(1-P)^k.\]
Make a conclusion.

\refstepcounter{chapter}
\setcounter{eqtn}{0}

\parbf{\ref{ex:bridge}.}
Show and use that any spanning tree of $G$ contains the bridge.

\parbf{\ref{ex:zig-zag}.}
Use induction on $n$ and/or mimic the proof of \ref{thm:fans} with the following analogs of $Z_n$:

\begin{Figure}
\centering
\includegraphics{mppics/pic-371}
\end{Figure}

\begin{wrapfigure}{r}{20 mm}
\vskip-4mm
\centering
\includegraphics{mppics/pic-39}
\end{wrapfigure}

\parbf{\ref{ex:ladder}.}
To construct the recursive relation, in addition to the ladders $L_n$, you will need two of its analogs --- $L_n'$ and $L_n''$, shown on the diagram.

\parbf{\ref{ex:wheel}.} Start with a spoke of the wheel.
The diagram will contain wheels, and several analogs of fans.

\refstepcounter{chapter}
\setcounter{eqtn}{0}

\parbf{\ref{ex:n(walks)}.}
Apply induction on $n$.

\parbf{\ref{ex:Kirchhoff-row}.}
Use property \ref{3} on page \pageref{3}.

\parbf{\ref{ex:minor>graph}.}
Revert the steps of the construction of the Kirchhoff minor.
(Since the Kirchhoff minor is $5{\times}5$, 
the graph should have 6 vertices.)

\parbf{\ref{ex:sum-kirchhoff}.}
Apply \ref{ex:Kirchhoff-row}.

\parbf{\ref{ex:K33W6Q3}.}
Apply the construction of Kirchhoff minor.

\parbf{\ref{ex:det}.}
Apply property \ref{3} twice and property \ref{2} (see page \pageref{3}).

\parbf{\ref{ex:s(Kp-e)}.}
We can assume that $e$ connects the last two vertices of $K_p$.
In this case
\[
M=\left(
\begin{matrix}
p{-}1&-1&\cdots&-1
\\
-1&\ddots&\ddots&\vdots
\\
\vdots&\ddots&p{-}1&-1
\\
-1&\cdots&-1&p{-}2
\end{matrix}
\right)
\]
is the Kirchhoff minor of $K_p-e$.
It remains to find $\det M$.

Follow the calculations in the proof of the Cayley formula.

\parbf{\ref{ex:s(Kmn)}.}
Note that
\[
M=
\left(
\begin{matrix}
3&0&0&0&-1&-1
\\
0&3&0&0&-1&-1
\\
0&0&3&0&-1&-1
\\
0&0&0&3&-1&-1
\\
-1&-1&-1&-1&4&0
\\
-1&-1&-1&-1&0&4
\end{matrix}
\right)\]
is a Kirchhoff minor of $K_{4,3}$.

First, show that $\det M=3^3\cdot 4^2$,
and then generalize the argument to $K_{m,n}$.

\refstepcounter{chapter}
\setcounter{eqtn}{0}

\parbf{\ref{ex:PG=PHPH}.}
Use that the subgrahs $H_1$ and $H_2$ can be colored independently.

\parbf{\ref{ex:PWn}.} Let $v$ be the center of $W_n$.
Suppose we have $x+1$ choices for color of $v$;
once this choice is made, the rest of $W_n-v$ has to be colored in the remaining $x$ colors.
Show and use that each such coloring corresponds to a coloring of $C_n$ in $x$ colors. 

\parbf{\ref{ex:PGpqn}.} 
By \ref{thm:chromatic-polynomial}, $P_G$ is a monic polynomial of degree $p$.
Show that $P_H(0)=0$ for any connected graph $H$.
Use \ref{ex:PG=PHPH} to show that for any $k<n$ the coefficient of $P_G$ in front of $x^k$ vanish.

To show that $a_{p-1}=q$,
use the deletion-minus-contraction formula in an induction on $q$.

\parbf{\ref{ex:P(tree)}.} The only-if part follows from \ref{ex:PTCpFnLn}\ref{ex:PTCpFnLn:tree}.
For the if part, apply \ref{ex:PGpqn} to show that $G$ is connected, has $p$ vertices and $p-1$ edges.
Make a conclusion.

\parbf{\ref{ex:chrom(K_p)}.} Apply the induction on $p$.

\parbf{\ref{ex:P=nonisom}.} Use \ref{ex:PTCpFnLn}\ref{ex:PTCpFnLn:tree}.

\parbf{\ref{ex:MG=MHMH}.} Use that the matchings in $H_1$ and $H_2$ can be chosen independently.

\parbf{\ref{ex:matchings}.} Use the definition of $M_G$.

\parbf{\ref{ex:deletion-deletion-total};}
\textit{\ref{ex:deletion-deletion}.}
Show and use that 
$$m_n(G)=m_n(G-e)+m_{n-1}(G-[e]).$$

\parit{\ref{ex:deletion-deletion-K}.}
Apply it $p$ times to all edges at one vertex of $K_p$.

\parbf{\ref{ex:SG}.} Use the definition of $S_G$.

\parbf{\ref{ex:S(Kp)}.}
Calculate $N=\tfrac{\partial^{k-1} }{\partial x_1^{k-1}}S_{K_n}(0,1,\dots,1)$ using the expression for $S_{K_n}$ given in Theorem~\ref{thm:spanning-tree-polynomial} and determine how much a tree with degree $d$ at the first vertex contributes to the value $N$.

\parbf{\ref{ex:S(Kmn)}.}
Modify the proof of Theorem~\ref{thm:spanning-tree-polynomial}.

\refstepcounter{chapter}
\setcounter{eqtn}{0}

\parbf{\ref{ex:B=xA}+\ref{ex:B=xA'}.} Apply the definition of exponential generating function.

\parbf{\ref{ex:exp(Fn)}.}
For \textit{\ref{ex:exp(Fn):F''}},
apply the definitions of Fibonacci numbers and exponential generating function.

For \textit{\ref{ex:exp(Fn):F(x)}}, solve the differential equation in \textit{\ref{ex:exp(Fn):F''}} and use that $f_0=f_1=1$.

For \textit{\ref{ex:exp(Fn):Binet}}, use the Taylor expansion 
\[e^x=1+\tfrac x{1!}+\tfrac {x^2}{2!}+\tfrac {x^3}{3!}+\dots\]

\parbf{\ref{ex:perfect-matching}.} 
Apply \ref{ex:deletion-deletion-total}\textit{\ref{ex:deletion-deletion-K}}.

\parbf{\ref{ex:an+nan-1}.}
Apply \ref{ex:deletion-deletion-total}\textit{\ref{ex:deletion-deletion-K}} for $x=1$.

\parbf{\ref{ex:ex:2-factor};} \textit{\ref{ex:2-factor:cn}.}
Enumerate the vertices of $K_n$.
Choose a cycle and list the numbers of the vertices in the order they appear on the cycle after $n$.
Count all possible orders.
Show and use that we have exactly two different orders for one cycle.

\refstepcounter{chapter}
\setcounter{eqtn}{0}

\parbf{\ref{ex:neighbor-trees}.}
Suppose $G$ is a minimal connected graph with two cycles;
that is, removing an edge or a vertex from $G$ makes it disconnected or reduces the number of cycles.

Show that $G$ contains two spanning trees that are not neighbors.
Use it to prove the general case.

\parbf{\ref{ex:w>2w}.} Denote the two weights by $\weight_1$ and $\weight_2$.
Suppose $T$ is a spanning tree and $T'$ is its neighbor. 
Show that 
\begin{align*}
\weight_1(T)&\le \weight_1(T')
\\
&\Updownarrow
\\
\weight_2(T)&\le \weight_2(T').
\end{align*}
Apply \ref{thm:mst-iff}.

\parbf{\ref{ex:PB}.} Apply \ref{thm:mst-iff}.

\parbf{\ref{ex:KPB}.}
Compare the number of edges that have to be checked in each algorithm.
Note that the organization of data might give an essential difference.
Think which steps could be done parallelly (say, on different computers).


\parbf{\ref{ex:deleting-algorithm}.}
Part \textit{\ref{ex:deleting-algorithm:a}} should be evident.
For part \textit{\ref{ex:deleting-algorithm:b}}, use \ref{thm:mst-iff}.

\parit{\ref{ex:deleting-algorithm:c}.}
Start to walk from an arbitrary vertex without coming back.
Stop at the end vertex or at the first vertex that was visited twice.
Decide what to do in each case and start over.

\refstepcounter{chapter}
\setcounter{eqtn}{0}

\parbf{\ref{ex:bigraph-matching}.} Show and use that $P$ has odd length.

\parbf{\ref{ex:1-factor};} \textit{\ref{ex:1-factor:a}.}
Note that a 1-factor is a matching.
Apply \ref{thm:marriage}.

\parit{\ref{ex:1-factor:b}.}
Remove the 1-factor provided by \textit{\ref{ex:1-factor:a}}. Apply \textit{\ref{ex:1-factor:a}} again and again.

\parit{Remark.}
If $r=2^n$ for an integer $n\ge 1$, then $G$ has an Euler's circuit. 
Note that the total number of edges in $G$ is even, so we can delete all odd edges from the circuit.
The obtained graph $G'$ is regular with degree $2^{n-1}$.
Repeating the described procedure recursively $n$ times, 
we will end up with a 1-factor of $G$.

There is a tricky way to make this idea work for arbitrary $r$, not necessarily a power of $2$; 
it was discovered by Noga Alon [see \ncite{alon} and also \ncite{kalai}]. 

\parbf{\ref{ex:no-1-factor}.}
Assume the graph, say $G$, has a matching.
Then the central vertex $v$ is matched with one of its three neighbors, say $w$.
Show and use that $G-v-w$ has even number of vertices in each connected component.

\parbf{\ref{ex:kids}.}
Consider the bigraph with vertices labeled by rows and countries.
Connect a row-vertex to a country-vertex if a kid from the corresponding county stands in the corresponding row.
Try to apply \ref{thm:marriage}.

\parbf{\ref{ex:sons(king)}.}
Consider the bigraph with vertices labeled by 23 old parts and 24 new parts.
Connect an old-part vertex to a new-part vertex if the corresponding parts overlap.
Try to apply \ref{thm:marriage}.

\parbf{\ref{ex:nxn-table}.}
Consider the bigraph with vertices labeled by rows and columns.
Connect a row vertex to a column vertex if a positive number stands in the common cell.
Try to apply \ref{thm:marriage}.

\parbf{\ref{ex:camel17}.}
Add $s$ boys that everyone likes.
Apply \ref{thm:marriage} and remove these $s$ boys.

\parbf{\ref{ex:two-paths}.}
Assume that two $M$-alternating paths starting from $l$ and $r$ can have a common vertex~$v$.
Show and use that there is an $M$-alternating paths starting from $l$ to $r$.

\parbf{\ref{ex:rooks}.} Consider the bigraph with a vertex for each column and row.
Connect a row vertex to a column vertex if their common cell is marked.
Apply \ref{thm:vertex-cover}.

\parbf{\ref{ex:min-cut-marriage}.} 
Let $m$ be the minimal number of edges one can remove from $\hat G$ 
so that there will be no directed path from $s$ to $t$.
Show that the assumptions in \ref{thm:marriage} imply that $m=|L|$.
Construct a bijection between directed paths from $s$ to $t$ in $\hat G$ 
corresponds to an edge in to an edge in $G$.
Apply \ref{thm:mincut}.

\refstepcounter{chapter}
\setcounter{eqtn}{0}

\parbf{\ref{ex:crossing1}.}
Attach a small handle to the sphere so that one edge can go thru it over the crossing.

\parbf{\ref{ex:toroidal-graphs}.}
\textit{(a)}
Modify the diagram for $K_7$.

\parit{(b)}+\textit{(c)} Start with the following square diagrams and draw the remaining edges.
\vskip2mm
\includegraphics{mppics/pic-119}
\hskip5mm
\includegraphics{mppics/pic-118}

\parbf{\ref{ex:nonplanar-toroidal}.}
Suppose that $\Delta$ is a nonsimple region in the drawing.
Cut $\Delta$ from the sphere and glue instead a disc or two to get a sphere.


\parbf{\ref{ex:toroidal-girth}.}
Show that $q\ge 2\cdot r$ and apply
\ref{thm:euler>=}.

\parbf{\ref{ex:K5-torus}.}
Assume it is possible.
Show that the drawing has 5 squares.
Try to glue 5 squares to each other side-to-side to obtain a torus. 

\parbf{\ref{ex:forbidden-minors}.}
Show that the answers are $K_{1,5}$ and $K_4-e$.

\parbf{\ref{ex:nontoroidal}.}
Formally, one needs to show that deletion and contraction of every edge makes the graph toroidal.
(Indeed, deleting a vertex produces a minor in a graph after deleting an edge.)
But the graph has many symmetries.
It reduces the number of cases to four:
two deletions and two constructions.

\parbf{\ref{ex:K6moebius}.} 
Make a Möbius strip; better use transparent paper.
Redraw the diagram on the strip and try to draw the remaining 6 edges.

\begin{Figure}
\centering
\includegraphics{mppics/pic-125}
\end{Figure}

\refstepcounter{chapter}
\setcounter{eqtn}{0}

\parbf{\ref{ex:rado-infty}.}
Assume it is finite,
then apply the definition including all the vertices in $V\cap W$. 


\parbf{\ref{ex:rado-diam}.}
Apply the definition for $V=\emptyset$ and a single vertex in $V$.
Conclude that the diameter is at least 2.

Choose two vertices $x$ and $y$.
Apply the definition for $V=\{x,y\}$ and $W=\emptyset$.
Conclude that there is a path of length 2 from $x$ to $y$,
and therefore the diameter is at most 2.

\parbf{\ref{ex:rado-partition}.}
Let $P$ and $Q$ be the induced subgraphs in the Rado graph $R$.
Assume $P$ is not Rado; that is, there is a pair of finite vertex sets $V$ and $W$ in $P$, such that any vertex $v$ in $R$ that meet the Rado property for $V$ and $W$ does not lie in $P$ (and therefore it lies in~$Q$).
Use $V$ and $W$ to show that $Q$ is Rado.

\parbf{\ref{ex:R-e-v-rev}+\ref{ex:rado-link}+\ref{ex:rado-costructive}.}
Check the definition in each case.
For \ref{ex:rado-costructive}, use the binary numeral system.


\parbf{\ref{ex:rado-path10}.}
Argue as in \ref{thm:rado-subgraph}.

\parbf{\ref{ex:rado-isom-generalization}.}
Find how to weaken the conditions at the odd steps of the construction.

\parbf{\ref{ex:Rv>w}.}
Apply \ref{thm:rado-isom}.

\parbf{\ref{ex:finite-subgraphs}.}
Show and use that the graphs have the same number of vertices and edges.

\parbf{\ref{ex:rado-radnom}.}
Choose two finite subsets of vertices $V$ and~$W$, and a vertex $v\notin V\cup W$.
Show that $v$ satisfies the definition with a fixed positive probability $\beta$.
Conclude that the definition holds with probability 1.

Finally, use that there is only countably many of choices for $V$ and $W$.

\parbf{\ref{ex:x+1}.}
Use the order 
\[1\prec 2\prec 4\prec 3\prec 6\prec 5\prec 8\prec 7\prec\dots\]

\parit{Remark.}
Proving that a given rewriting system is terminating might be difficult.
For example,
it is unknown if the following system is terminating:
the vertex set formed by all positive integers and the rules $x\to \tfrac x2$ if $x$ is even and $x\to 3\cdot x+1$ if $x>1$ is add.
This is the so-called $(3\cdot x+1)$-problem.

\parbf{\ref{ex:balls}.}
Consider the following order on pairs of integers:
$(m,n)\succ(m',n')$ in two cases: if $m>m'$, or if $m=m'$ and $n>n'$.
(In fact it is the lexicographic or on pairs.)
Apply this order assuming that the pair $(m,n)$ describes a box with $m$ white balls and $n$ black balls.

\parbf{\ref{ex:ab>ba,aba>}.}
Use the shortlex order.

\parbf{\ref{ex:complete}.}
\textit{\ref{ex:complete:a}} It is terminating but not confluent.

\parit{Remark.} Note that the systems in \ref{em:baaba>a}, \ref{em:baaba>a,baaa>aaba}, and \textit{\ref{ex:complete:a}} have the same equivalence relation $\sim$.
The Knuth--Bendix procedure that led us from \ref{em:baaba>a} to \ref{em:baaba>a,baaa>aaba} lasts indefinitely, but all new relations that arise have the form
\[bb(ab)^{m}aaa\to a(ba)^m.\]
All of them, taken at $m=0,1,2,\dots$ together with the relation $aaba\to baaa$, form a complete system.

\begin{wrapfigure}{r}{20 mm}
\vskip-2mm
\centering
\begin{tikzpicture}[node distance=1.3cm, auto]
\node (x) {$cba$};
\node (y) [right of=x] {$bca$};
\node (zz) [below of=x, node distance=.9cm] {$cab$};
\node (v) [below of=y, node distance=.9cm] {$abc$};
\draw[->] (x) to (zz);
\draw[->] (x) to (y);
\draw[->,decorate,decoration=zigzag] (y) to (v);
\draw[->,decorate,decoration=zigzag] (zz) to (v);
\end{tikzpicture}
\end{wrapfigure}

\parit{\ref{ex:complete:b}}
Use lexicographic order to show that the system is terminating.
For confluence, check the following diagram and show that this is the only thing to check.


\parit{\ref{ex:complete:c}}
Note that each rule shortens the word.
Use it to show that the system is terminating.

There are 6 possible overlaps here, but they are all of the same type.
It is sufficient to consider one word $xXx$.

\parit{Remark.}
А discerning reader could notice that this example describes a free group of rank 3;
capital letters correspond to inverses. 

\spell{\end{multicols}}{}
\newpage
