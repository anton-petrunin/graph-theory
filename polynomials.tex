\chapter{Polynomials}

Counting problems often lead to an organized collection of numbers.
Sometimes it is convenient to consider a polynomial with these numbers as coefficients.
If it is done in a smart way, then the algebraic structure of the obtained polynomial reflects the original combinatorial structure of the graph.

\section*{Chromatic polynomial}

Denote by $P_G(x)$ the number of different colorings of the graph $G$ in $x$ colors such that the ends of each edge get different colors.

\begin{thm}{Exercise}
Assume that a graph $G$ has exactly two connected components $H_1$ and $H_2$.
Show that 
\[P_G(x)=P_{H_1}(x)\cdot P_{H_2}(x)\]
for any $x$.
\end{thm}

\begin{thm}{Exercise}
Show that for any integer $n\ge 3$,
\[P_{W_n}(x+1)=(x+1)\cdot P_{C_n}(x),\]
where $W_n$ denotes a wheel with $n$ spokes and $C_n$ is a cycle of length~$n$.
\end{thm}

\begin{thm}{Deletion-minus-contraction formula}\label{thm:deletion-minus-contraction}\index{deletion-minus-contraction formula}
For any edge $e$ in the pseudograph $G$ we have
\[P_G(x)=P_{G\backslash e}(x)-P_{G/e}(x).\eqlbl{eq:deletion-minus-contraction}\]
\end{thm}

\parit{Proof.}
The valid colorings of $G\backslash e$ can be divided into two groups: 
(1) those where the ends of the edge $e$ get different colors --- these remain to be valid colorings of $G$ and (2) those where the ends of $e$ get the same color --- each of such colorings corresponds to a unique coloring of $G/e$.
Hence
\[P_{G\backslash e}(x)=P_G(x)+P_{G/e}(x),\]
which is equivalent to the deletion-minus-contraction formula \ref{eq:deletion-minus-contraction}.
\qeds

Note that if the pseudograph $G$ has loops, then $P_G(x)=0$ for any $x$.
Indeed, in a valid coloring the ends of a loop should get different colors, which is impossible.

The latter can also be proved using the deletion-minus-contraction formula.
Indeed, if $e$ is a loop in $G$, then $G/e=G\backslash e$.
Therefore $P_{G\backslash e}(x)\z=P_{G/e}(x)$ and
\[P_G(x)=P_{G\backslash e}(x)-P_{G/e}(x) =0.\]

Similarly, removing a parallel edge from a pseudograph $G$ does not change the value $P_G(x)$ for any $x$.
Indeed, if $e$ is an edge of $G$ which has a parallel edge $f$, then in $G/e$ the edge $f$ becomes a loop.
Therefore $P_{G/e}(x)=0$ for any $x$ and by the deletion-minus-contraction formula we get that
\[P_G(x)=P_{G\backslash e}(x).\]
The same identity can be seen directly --- any admissible coloring of $G\backslash e$ is also admissible in $G$ --- since the ends of $f$ get different colors, so does $e$. 

Summarizing the discussion above:
the problem of finding $P_G(x)$ for a pseudograph $G$ can be reduced to the case when $G$ is a graph; that is, $G$ has no loops and no parallel edges.
Indeed, if $G$ has a loop, then $P_G(x)=0$ for all $x$.
Further, removing one of the parallel edges from $G$ does not change $P_G(x)$.

Recall that polynomial $P$ of $x$ is an expression of the following type
\[P(x)=a_0+a_1\cdot x+\dots+a_n\cdot x^n,\]
with constants $a_0,\dots, a_n$, which are called {}\emph{coefficients} of the polynomial.
If $a_n\ne 0$, it is called the \index{leading coefficient}\emph{leading coefficient} of $P$;
in this case $n$ is the degree of $P$.
If the leading coefficient is 1, then the polynomial is called \index{monic polynomial}\emph{monic}.

\begin{thm}{Theorem}\label{thm:chromatic-polynomial}
Let $G$ be a pseudograph with $p$ vertexes.
Then $P_G(x)$ is a polynomial with integer coefficients.

Moreover, if $G$ has a loop, then $P_G(x)\equiv 0$;
otherwise $P_G(x)$ is monic and has degree $p$.
\end{thm}

Based on this result we can call $P_G(x)$ the \index{chromatic polynomial}\emph{chromatic polynomial} of the graph~$G$.
The deletion-minus-contraction formula will play the central role in the proof.

\parit{Proof.}
As usual, denote by $p$ and $q$ the number of vertexes and edges in $G$.
To prove the first part, we will use the induction on $q$.

As the base case, consider the null graph $N_p$;
that is, the graph with $p$ vertexes and no edges.
Since $N_p$ has no edges, any coloring of $N_p$ is admissible.
We have $x$ choices for each of $n$ vertexes therefore
\[P_{N_p}(x)=x^p.\]
In particular, the function $x\mapsto P_{N_p}(x)$ is given by a monic polynomial of degree $p$ with integer coefficients.

Assume that the first statement holds for all pseudographs with at most $q-1$ edges.
Fix a pseudograph $G$ with $q$ edges. 
Applying the deletion-minus-contraction formula for some edge $e$ in $G$, we get that
\[P_G(x)=P_{G\backslash e}(x)-P_{G/e}(x).\eqlbl{eq:deleting-contracting}\]
Note that the pseudographs $G\backslash e$ and $G/e$ have $q-1$ edges.
By the induction hypothesis, $P_{G\backslash e}(x)$ and $P_{G/e}(x)$ are polynomials with integer coefficients.
Hence \ref{eq:deleting-contracting} implies the same for $P_G(x)$.

If $G$ has a loop, then $P_G(x)=0$, as $G$ has no valid colorings.
It remains to show that if $G$ has no loops, then $P_G(x)$ is a monic polynomial of degree $p$.

Assume that the statement holds for any multigraph $G$ with at most $q-1$ edges and at most $p$ vertexes.

Fix a multigraph $G$ with $p$ vertexes and $q$ edges.
Note that $G\backslash e$ is a multigraph with $p$ vertexes and $q-1$ edges.
By the assumption, its chromatic polynomial $P_{G\backslash e}$ is monic of degree $p$.

Further the pseudograph $G/e$ has $p-1$ vertexes,
and its chromatic polynomial $P_{G/e}$ either vanishes or it has degree $p-1$.
In both cases the difference $P_{G\backslash e}-P_{G/e}$ is a monic polynomial of degree~$p$.
It remains to apply \ref{eq:deleting-contracting}.
\qeds

\begin{thm}{Advanced exercise}
Let $G$ be a graph with $p$ vertexes and $q$ edges.
Show that the coefficient in front of $x^{p-1}$ of its chromatic polynomial $P_G(x)$ equals to $(-q)$.
\end{thm}

\parit{Hint:} Apply induction on $q$ and use the deletion-minus-contraction formula the same way as in the proof of the theorem.

\begin{thm}{Exercise}
Use induction and the deletion-minus-contraction formula to show that 
\begin{enumerate}[(a)]
\item $P_{T}(x)=x\cdot(x-1)^q$ for any tree $T$ with $q$ edges;
\item $P_{C_p}(x)=(x-1)^p+(-1)^p\cdot(x-1)$ for the cycle $C_p$ of length $p$.
\end{enumerate}
\end{thm}


\begin{thm}{Exercise} Show that graph $G$ is a tree if and only if \[P_G(x)= x\cdot(x-1)^{p-1}\] for some positive integer $p$.
\end{thm}

\begin{thm}{Exercise}\label{ex:chrom(K_p)}
Show that 
\[P_{K_p}(x)=x\cdot(x-1)\cdots(x-p+1).\]

\end{thm}

\parbf{Remark.}
Note that for any graph $G$ with $p$ vertexes we have
\[P_{K_p}(x)\le P_G(x)\le P_{N_p}(x)\]
for any $x$.
Since both polynomials
\begin{align*}
P_{K_p}(x)&=x\cdot(x-1)\cdots(x-p+1),&&
\text{and}
&
P_{N_p}(x)&=x^p,
\end{align*}
are monic of degree $p$,
it follows that so is $P_G$.

Hence Exercise~\ref{ex:chrom(K_p)} leads to an alternative way to prove the second statement in Theorem~\ref{thm:chromatic-polynomial}.

\begin{thm}{Exercise}
Construct a pair of nonisomorphic graphs with equal chromatic polynomials.
\end{thm}

\section*{Matching polynomial}

Recall that a \index{matching}\emph{matching} in a graph is a set of edges without common vertexes.

Given an integer $n\ge0$,
denote by $m_n=m_n(G)$ the number of matchings with $n$ edges in the graph $G$.

Note that for a graph $G$ with $p$ vertexes and $q$ edges, we have 
$m_0(G)=1$, 
$m_1(G)=q$, 
and if $2\cdot n>p$, then $m_n(G)=0$.
The maximal integer $k$ such that $m_k(G)\ne0$ is called the \index{matching number}\emph{matching number} of $G$.
The expression 
\[M_G(x)=m_0+m_1\cdot x+\dots +m_k\cdot x^k\]
is called the \index{matching polynomial}\emph{matching polynomial} of $G$.

The matching polynomial $M_G(x)$ gives a convenient way to work with all the numbers $m_n(G)$ simultaneously.
The degree of $M_G(x)$ is the matching number of $G$ and
the total number of matching in $G$ is its value at $1$:  
\[M_G(1)=m_0+m_1+\dots +m_k.\]

\begin{thm}{Exercise}
Show that the values
\[\tfrac12\cdot[M_G(1)+ M_G(-1)]\quad\text{and}\quad\tfrac12\cdot[M_G(1)- M_G(-1)]\]
equal to the number of matchings with even and odd number of edges correspondingly.
\end{thm}

{

\begin{wrapfigure}{r}{33 mm}
\vskip-8mm
\centering
\includegraphics{mppics/pic-51}
\vskip-0mm
\end{wrapfigure}

Assume $e$ is an edge in a graph $G$.
Recall that the graph $G\backslash e$ is obtained by deleting $e$ from $G$.
Let us denote by $G\backslash \bar e$ the graph obtained by deleting the vertexes of $e$ with all their edges from $G$;
that is, if $e$ connects two vertexes $v$ and $w$, then 
\[G\backslash \bar e=G\backslash \{v,w\}.\]

}

The following exercise is analogous to the deletion-contraction formulas \ref{thm:deletion-minus-contraction} and \ref{thm:deletion-plus-contraction}.

\begin{thm}{Exercise}\label{ex:deletion-deletion-total}
Let $G$ be a graph.
\begin{enumerate}[(a)]
\item\label{ex:deletion-deletion} Show that
\[M_G(x)=M_{G\backslash e}(x)+x\cdot M_{G\backslash \bar e}(x)\]
for any edge $e$ in $G$.

\item\label{ex:deletion-deletion-K} Use part \ref{ex:deletion-deletion} to show that the matching polynomials of complete graphs satisfy the following recursive relation:
\[M_{K_{p+1}}(x)=M_{K_{p}}(x)+p\cdot x\cdot M_{K_{p-1}}(x).\]

\item Use \ref{ex:deletion-deletion-K} to calculate $M_{K_n}(x)$ for $1\le n\le 6$.
\end{enumerate}

\end{thm}

\section*{Spanning-tree polynomial}

Consider a connected graph $G$ with $p$ vertexes;
assume $p\ge 2$.

Let us prepare independent variables $x_1,\dots,x_p$, one for each vertex of $G$.
For each spanning tree $T$ in $G$ consider the monomial 
\[x_1^{d_1-1}\cdots x_p^{d_p-1},\]
where $d_i$ denotes the degree of the $i$-th vertex in $T$.

The tree $T$ has $p-1$ edges and therefore 
$d_1+\dots+d_p=2\cdot(p-1)$.
It follows that the total degrees of the monomial is $p-2$.

The sum of these monomials is a polynomial of degree $p-2$ of $p$ variables $x_1,\dots, x_p$.
This polynomial will be called the \index{spanning-tree polynomial}\emph{spanning-tree polynomial}of $G$ and it will be denoted by 
$S_G(x_1,\dots,x_p)$.

Note that the total number of spanning trees in $G$ equals to the value
$S_G(1,\dots,1)$.
The following exercise shows that,
the polynomial $S_G$ keeps a lot more information about spanning trees in $G$.

%Abusing notation slightly, we will use the variables $x_1,\dots,x_p$ as label of the corresponding vertex in $G$; for instance, we can say ``$x_2$ has degree 5 in $G$'' instead of ``the second vertex has degree 5 in $G$''.   

\begin{thm}{Exercise}
Let $S_G(x_1,\dots,x_p)$ be the spanning-tree polynomial of a graph $G$.
Show the following:
\begin{enumerate}[(a)]
\item The value $S_G(0,1,\dots,1)$ can be interpreted as the number of spanning trees with the leaf the first vertex.
\item The coefficient of $S_G$ in front of $x_1\cdots x_{p-2}$ equals to the number of paths of length $p-1$ connecting $(p-1)$-th and $p$-th vertexes.
\item The partial derivative
\[\tfrac{\partial}{\partial x_1}S_G(0,1,\dots,1)\]
is the numbers of spanning trees in $G$ with degree 2 at the first vertex.
\item The values 
\[\tfrac12\cdot\left[S_G(1,1,\dots,1)\pm S_G(-1,1,\dots,1)\right]\]
are the numbers of spanning trees in $G$ with odd or correspondingly even degree at the first vertex.
\end{enumerate}
\end{thm}

The following theorem generalizes the Cayley formula (\ref{thm:cayley}).

\begin{thm}{Theorem}\label{thm:spanning-tree-polynomial}
\[S_{K_p}(x_1,\dots,x_p)=(x_1+\dots +x_p)^{p-2},\]
where $K_p$ is the complete graph with $p\ge 2$ vertexes.
In particular, $s(K_p)=p^{p-2}$;
that is, the number of spanning trees in $K_p$ is $p^{p-2}$.
\end{thm}

\parit{Proof.}
Let us apply induction on $p$;
the base case $p=2$ is evident.

Assume that the statement holds for $p-1$; that is,
\[S_{K_{p-1}}(x_1,\dots,x_{p-1})=(x_1+\dots+x_{p-1})^{p-3}.
\eqlbl{eq:S-K(n-1)}\]
We need to show that 
\[S_{K_p}(x_1,\dots,x_p)=(x_1+\dots+x_p)^{p-2}.\]

Note that $S_{K_p}$ is a {}\emph{homogeneous} {}\emph{symmetric} polynomial of degree $p-2$.
That is, each monomial in $S_{K_{p}}$ has degree $p-2$ and any permutation of values $x_1,\dots, x_p$ does not change the value $S_G(x_1,\dots,x_p)$.

Therefore it suffices to prove that 
\begin{clm}{}\label{clm:sum-up}
all monomials in $S_{K_p}$ without $x_p$ sum up to $(x_1+\dots+x_{p-1})^{p-2}$. 
\end{clm}

Note that each of these monomials corresponds to a spanning tree $T$ with $d_p=1$;
that is, the vertex $x_p$ is a leaf of $T$.
In other words, the tree $T$ is obtained from another tree $T'$ with the vertexes $x_1,\dots,x_{p-1}$ 
by adding an edge from $x_p$ to $x_i$ for some $i<p$.
Note that the monomial in $S_{K_p}$ that corresponds to $T$ equals 
to the product of $x_i$ times the monomial in $S_{K_{p-1}}$
that corresponds to $T'$.

To get the sum of all monomials in $S_{K_p}$ without $x_p$, we need to sum up these products for all $i<p$ and all the monomials in $S_{K_{p-1}}$; this way we get 
\[S_{K_{p-1}}(x_1,\dots,x_{p-1})\cdot(x_1+\dots+x_{p-1}).\]
By \ref{eq:S-K(n-1)}, the latter equals to
\[(x_1+\dots+x_{p-1})^{p-2}\]
which proves \ref{clm:sum-up}.

To prove the last statement, it remains to note that 
\[s(G)=S_G(1,\dots,1)=(1+\dots+1)^{p-2}=p^{p-2}.\]
\qedsf

\begin{thm}{Exercise}
Assume that the vertexes of the left part of $K_{m,n}$ have corresponding variables $x_1,\dots,x_m$ and the vertexes in the right part have corresponding variables $y_1,\dots,y_n$. 
Show that
\[S_{K_{m,n}}(x_1,\dots,x_m,y_1,\dots,y_n)=(x_1+\dots +x_m)^{n-1}\cdot(y_1+\dots +y_n)^{m-1}.\]
Conclude that $s(K_{m,n})=m^{n-1}\cdot n^{m-1}$.

\end{thm}

\parit{Hint:} Modify the proof of Theorem \ref{thm:spanning-tree-polynomial}.

\section*{Remarks}

A very good expository paper on chromatic polynomials is written by
Ronald Read \cite[see][]{read}. 
Matching polynomials are discussed in a paper of Christopher Godsil and Ivan Gutman \cite[see][]{godsil-gutman}.

Our discussion of spanning-tree polynomials is based on a modification of Fedor Petrov \cite{petrov} of the original proof of Arthur Cayley~\cite{cayley}. 

Generating functions discussed in Appendix~\ref{Generating functions} give connections between graph theory and power series; it is more challenging, but worth to learn.   
