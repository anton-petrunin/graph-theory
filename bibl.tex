\begin{thebibliography}{52}

\bibitem{aigner-ziegler} M. Aigner, G. M. Ziegler,  \emph{Proofs from The Book.} 2014.

\bibitem{alon} N. Alon,
\emph{A simple algorithm for edge-coloring bipartite multigraphs.}
Inform. Process. Lett. 85 (2003), no. 6, 301--302. 

\bibitem{cameron} P. Cameron,   \emph{The random graph.} \texttt{arXiv:1301.7544}.

\bibitem{cayley} 
A. Cayley, \emph{A theorem on trees.} Quart. J. Pure Appl. Math. 23 (1889) 376--378.


\bibitem{egervary}E. Egerv\'ary, 
\emph{Matrixok Kombinatorius Tulajdons\'agairol}
Matematikai \'es Fizikai Lapok. 38 (1931), 16--28.

\bibitem{EFS} P. Elias, A. Feinstein, and C. Shannon. 
\emph{A note on the maximum flow through a network.}
IRE Transactions on Information Theory 2, no. 4 (1956), 117--119.

\bibitem{evnin} \begin{otherlanguage}{russian}
А.~Ю.~Эвнин,
\emph{Вокруг теоремы Холла.} Матем. обр. 3(34) 2005, 2---23.
 \end{otherlanguage}

\bibitem{ford-fulkerson} L. Ford, D. Fulkerson. 
\emph{Maximal flow through a network.} 
Canadian journal of Mathematics 8 (1956) no. 3, 399--404.

\bibitem{gardiner} M. Gardner, \emph{The 2nd Scientific American Book of Mathematical Puzzles and Diversions.}

\bibitem{godsil-gutman}
C. D. Godsil, I. Gutman, 
\emph{On the theory of the matching polynomial.}
J. Graph Theory 5 (1981), no. 2, 137–144.

\bibitem{hall} 
P. Hall, 
\emph{On Representatives of Subsets}, 
J. London Math. Soc., vol. 10 (1935), 26--30.

\bibitem{harary-palmer} F. Harary, E. Palmer, 
\emph{Graphical enumeration.} 1973.

\bibitem{pearls} N. Hartsfield and  G. Ringel, 
\emph{Pearls in graph theory: a comprehensive introduction.} 2013.

\bibitem{17-camels} G. Kalai, 
\emph{The seventeen camels riddle, and Noga Alon’s camel proof and algorithms.}
\texttt{https://gilkalai.wordpress.com/} 2017/02/16/ 

\bibitem{lovasz} L. Lov\'asz, L\'aszl\'o \emph{Combinatorial problems and exercises.} Corrected reprint of the 1993 second edition. AMS Chelsea Publishing, Providence, RI, 2007. 642 pp. ISBN: 978-0-8218-4262-1

\bibitem{knut} R. L. Graham, D. E. Knuth, O. Patashnik, 
\emph{Concrete mathematics. A foundation for computer science.}
1994.

\bibitem{jordan} C. Jordan, 
\emph{Calculus of finite differences.}
1939

\bibitem{konig} D. K\H{o}nig, \emph{Gr\'{a}fok \'{e}s m\'{a}trixok}, Matematikai és Fizikai Lapok, 38 (1931): 116--119.

\bibitem{levi} M. Levi,
An Electrician’s (or a plumber’s)
proof of Euler’s polyhedral formula,
\emph{SIAM News} 50, no. 4, May 2017.

\bibitem{petrov} F. Petrov, Generating function in graph theory,
\emph{MathOverflow}
\texttt{https://mathoverflow.net/q/287767} (version: 2017-12-05)

\bibitem{read} R. Read, 
An introduction to chromatic polynomials.
\emph{J. Combinatorial Theory} 4 1968 52--71.

\bibitem{haghighi-bibak} M. H. Shirdareh Haghighi, Kh. Bibak, 
Recursive relations for the number of spanning trees. 
\emph{Appl. Math. Sci. (Ruse)} 3 (2009), no. 45--48, 2263--2269.
\end{thebibliography}

