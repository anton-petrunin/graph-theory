\chapter{Generating functions}\label{Generating functions}
 
For this chapter, the reader has to be familiar with power series.


\section*{Exponential generating functions}

The power series 
\[A(x)=a_0+a_1\cdot x+\tfrac12\cdot a_2\cdot x^2+\dots+\tfrac1{n!}\cdot a_n\cdot x^n+\dots\]
is called \index{exponential generating function}\emph{exponential generating function} of the sequence $a_0,a_1,\dots$.

If the series $A(x)$ converges in some neighborhood of zero, then it defines a function which remembers all information of the sequence~$a_n$.
The latter follows since 
\[A^{(n)}(0)=a_n;\eqlbl{eq:An=an}\]
that is, the $n$-th derivative of $A(x)$ at $0$ equals to $a_n$.

However, without assuming the convergence, we can treat $A(x)$ as a formal power series.
We are about to describe how to do addition, multiplying, taking derivative and so on.

\parbf{Sum and product.}
Consider two exponential generating functions
\begin{align*}
A(x)&=a_0+a_1\cdot x+\tfrac12\cdot a_2\cdot x^2+\tfrac16\cdot a_3\cdot x^3+\dots
\\
B(x)&=b_0+b_1\cdot x+\tfrac12\cdot b_2\cdot x^2+\tfrac16\cdot b_3\cdot x^3+\dots
\end{align*}
We will write 
\[S(x)=A(x)+B(x),\quad 
P(x)=A(x)\cdot B(x)\]
if the power series $S(x)$ and $P(x)$ are obtained from $A(x)$ and $B(x)$ by opening the parentheses these formulas and combining similar terms.

It is straightforward to check that $S(x)$ is the exponential generating function for the sequence  
\begin{align*}
s_0&=a_0+b_0,
\\
s_1&=a_1+b_1,
\\
&\dots
\\
s_n&=a_n+b_n,
\\
&\dots
\end{align*}
The product $P(x)$ is also exponential generating function for the sequence
\[
\begin{aligned}
p_0&=a_0\cdot b_0,
\\
p_1&=a_0\cdot b_1+a_1\cdot b_0,
\\
p_2&=a_0\cdot b_2+2\cdot a_1\cdot b_1+a_2\cdot b_0,
\\
p_3&=a_0\cdot b_3+3\cdot a_1\cdot b_2+3\cdot a_2\cdot b_1+a_3\cdot b_0,
\\
&\dots
\\
p_n&=\sum_{i=0}^n\tbinom in\cdot a_i\cdot b_{n-i}.
\end{aligned}
\eqlbl{eq:multiplication}
\]

\begin{thm}{Exercise}
Assume $A(x)$ exponential generating function of the sequence $a_0,a_1,\dots$.
Show that $B(x)=x\cdot A(x)$ corresponds to the sequence $b_n=n\cdot a_{n-1}$.
\end{thm}


\parbf{Composition.}
Once we define addition and multiplication of power series we can also plug in one power series in another.
For example, if $a_0=0$ the expression 
\[E(x)=e^{A(x)}\] is another power series which is obtained by plugging $A(x)$ instead of $x$ in the power series of exponent:
\[e^x=1+x+\tfrac12\cdot x^2+\tfrac16\cdot x^3+\dots\]
It is harder to express the sequence $(e_n)$ corresponding to $E(x)$ in terms of $a_n$, but it is easy to find first few terms.
Since we assume $a_0=0$, we have
\begin{align*}
e_0&=1,
\\
e_1&=a_1,
\\
e_2&=a_2+2\cdot a_1^2,
\\
e_3&=a_3+6\cdot a_1\cdot a_2,
\\
&\dots
\end{align*}

\parbf{Derivative.}
The derivative of $A(x)$ is defined as the following formal power series 
\[A'(x)=a_1+a_2\cdot x+\tfrac12\cdot a_3\cdot x^2+\dots+\tfrac1{n!}\cdot a_{n+1}\cdot x^n+\dots\]
Note that $A'(x)$ coincides with the ordinary derivative of $A(x)$ if the latter converges.

Note that $A'(x)$ is the exponential generating function of the sequence 
\[a_1,a_2,a_3,\dots\]
obtained from the original sequence 
\[a_0,a_1,a_2,\dots\]
by deleting the zero-term and shifting the indexes by 1.


\begin{thm}{Exercise}
Let $A(x)$ be the exponential generating function of the sequence $a_0,a_1,a_2\dots$.
Describe the sequence $b_n$ for which 
\[B(x)=x\cdot A'(x).\]
is the exponential generating function.
\end{thm}

\parbf{Calculus.}
If $A(x)$ converges and
\[E(x)=e^{A(x)},\] 
then we have 
\[\ln E(x)=A(x).\]
Also taking derivative of $E(x)=e^{A(x)}$ we get that
\begin{align*}
E'(x)
&=e^{A(x)}\cdot A'(x)=
\\
&= E(x)\cdot A'(x).
\end{align*}


These identities have perfect meaning in terms of formal power series
and they still hold without assuming the convergence.
We will not prove it formally, but this is not hard.

\section*{Fibonacci numbers}

Recall that Fibonacci numbers $f_n$ are defined using the recursive identity 
$f_{n+1}=f_n+f_{n-1}$
with $f_0=0$, $f_1=1$.

\begin{thm}{Exercise}
Let $F(x)$ be the exponential generating function of Fibonacci numbers $f_n$.
\begin{enumerate}[(a)]
\item Show that it satisfies the following differential equation
\[F''(x)=F(x)+F'(x).\]
\item Conclude that 
\[F(x)=\frac{1}{\sqrt5}\cdot\left(e^{\frac{1+\sqrt{5}}{2}\cdot x}- e^{\frac{1-\sqrt{5}}{2}\cdot x}\right).\]
\item Use the identity \ref{eq:An=an} to derive 
\[f_n=\tfrac{1}{\sqrt5}\cdot\left((\tfrac{1+\sqrt{5}}{2})^n-(\tfrac{1+\sqrt{5}}{2})^n\right).\]
(This is the so called \index{Binet's formula}\emph{Binet's formula}.) 
\end{enumerate}

\end{thm}


\section*{Exponential formula}

Fix a set of graphs $\mathcal{S}$.
Denote by $c_n=c_n(\mathcal{S})$ the number of spanning subgraphs of $K_n$ isomorphic to one of the graphs in $\mathcal{S}$. 
Let \[C(x)=C_{\mathcal{S}}(x)\] be the exponential generating function of the sequence~$c_n$. 


\begin{thm}{Theorem}\label{thm:exp-formula}
Let $\mathcal{S}$ be a set of connected graphs. 

\begin{enumerate}[(a)]
\item\label{thm:exp-formula:Wk} Fix a positive integer $k$ and denote by $w_n$ the number of spanning subgraphs of $K_n$ which have exactly $k$ connected components and each connected component is  isomorphic to one of the graphs in $\mathcal{S}$.
Then
\[W_k(x)=\tfrac1{k!}\cdot C_{\mathcal{S}}(x)^k,\]
where $W_k(x)$ is the exponential generating function of the sequence $w_n$.

\item\label{thm:exp-formula:all} Denote by $a_n$ the number of {}\emph{all} spanning subgraphs of $K_n$ such that each connected component of it is from the class and let $A(x)$ be the corresponding exponential generating function.
Then
\[1+A(x)=e^{C_{\mathcal{S}}(x)}.\]
\end{enumerate}

\end{thm}

Taking logarithm and derivative of the formula in \ref{thm:exp-formula:all},
we get the following:

\begin{thm}{Corollary}\label{cor:exp-formula}
Assume $A(x)$ and $C(x)$ as in Theorem~\ref{thm:exp-formula}\ref{thm:exp-formula:all}.
Then
\[\ln [1+A(x)]=C(x)\quad\text{and}\quad A'(x)=[1+A(x)]\cdot C'(x).\]
\end{thm}

The second formula in this corollary provides a recursive formula for the corresponding sequences which will be important latter.

\parit{Proof; \ref{thm:exp-formula:Wk}.}
Denote by $v_n$ the number of spanning subgraphs of $K_n$ which have $k$ ordered connected components and each connected component is isomorphic to one of the graphs in $\mathcal{S}$.
Let $V_k(x)$ be the corresponding generating function.

Note that for each graph as above
there are $k!$ ways to order its $k$ components.
Therefore $w_n=\tfrac1{k!}\cdot v_n$ for any $n$ and
\[W_k(x)=\tfrac1{k!}\cdot V(x).\]
Hence it is sufficient to show that 
\[V_k(x)=C(x)^k.
\eqlbl{eq:V=Ck}\]

To prove the latter identity, we apply induction on $k$ and the multiplication formula \ref{eq:multiplication} for exponential generating functions.
The base case $k=1$ is evident.

Assuming that the identity \ref{eq:V=Ck} holds for $k$;
we need to show that 
\[V_{k+1}=V_k(x)\cdot C(x).\eqlbl{eq:VkC}\]

Assume that a spanning graph with ordered $k+1$ connected components of $K_n$ is given.
Denote by $m$ the number of vertexes in the first $k$ components.
There are $\tbinom mn$ ways to choose these vertexes among $n$ vertexes of $K_n$ and for each choice we have
$v_m$ ways to choose spanning subgraph with $k$ components in it;
the last component has $m-n$ vertexes and we have $c_{n-m}$ ways to choose a subgraph from $\mathcal{S}$.
All together we get that
\[\tbinom mn\cdot v_m\cdot c_n.\]
Summing it up for all $m$ we get the multiplication formula \ref{eq:multiplication} for exponential generating functions.
Hence \ref{eq:VkC} follows.

\parit{\ref{thm:exp-formula:all}.}
To count all graphs we need to add number of spanning graphs for all number of components;
that is,
\begin{align*}A(x)&=W_1(x)+W_2(x)+\dots=
\intertext{Applying to part \ref{thm:exp-formula:Wk}, we can continue}
&=C(x)+\tfrac12\cdot C(x)^2+\tfrac16\cdot C(x)^3+\dots=
\\
&=e^{C(x)}-1.
\end{align*}
The last equality follows since 
\[e^x=1+x+\tfrac12\cdot x^2+\tfrac16\cdot x^3+\dots\]
Hence the result.
\qeds

\section*{Sample applications}

The following calculations can be done without using Theorem~\ref{thm:exp-formula};
this theorem only provides a general point of view to these problems.

\parbf{Perfect matchings.}
Recall that a perfect matching is 1-factor of the graph. 
In other words, it is a set of isolated edges which cover all the vertexes.
Note that if a graph admits a perfect matching then the number of its vertexes is even.

Recall that \index{double factorial}\emph{double factorial} is the product of all the integers from $1$ up to some non-negative integer $n$ that have the same parity (odd or even) as $n$;
the double factorial of $n$ is denoted by $n!!$.
For example, 
\[9!! = 9\cdot 7 \cdot 5 \cdot 3 \cdot  1 = 945
\quad\text{and}\quad
10!!=10\cdot8\cdot6\cdot4\cdot2=3840.
\]

\begin{thm}{Exercise}\label{ex:matching}
Let $a_n$ denotes the number of perfect matching in $K_n$.
Show that 
\begin{enumerate}[(a)]
 \item $a_2=1$;
 \item $a_n=0$ for odd $n$;
 \item\label{ex:matching:recursion} $a_{n+1}=n\cdot a_{n-1}$ for any integer $n\ge 2$.
 \item\label{ex:matching:n!!} 
 Conclude that $a_n=0$ and $a_{n+1}=n!!$ for odd $n$.
\end{enumerate}

\end{thm}

Now we give a more complicated proof of Exercise~\ref{ex:matching}\ref{ex:matching:n!!}.

\begin{thm}{Problem}
Use Theorem~\ref{thm:exp-formula} to show that number of perfect matching in $K_{2\cdot n}$ is $(2\cdot n-1)!!$.
\end{thm}

\parit{Solution.} 
Denote by $a_n$ the number of perfect matching in $K_{n}$ and let $A(x)$ be the corresponding exponential generating function.

Note that a perfect matching can be defined as a spanning subgraph such that each connected component is isomorphic to $K_2$.
So we can apply the formula in Theorem~\ref{thm:exp-formula} for the set $\mathcal{S}$ consisting of only one graph $K_2$.

Note that if $K_n$ contains a spanning subgraph isomorphic to $K_2$,
then $n=2$.
It follows that $c_2(\mathcal{S})=1$ and $c_n(\mathcal{S})=0$ for $n\ne 2$.
Therefore 
\[C(x)=C_{\mathcal{S}}(x)=\tfrac12\cdot x^2.\]

By Theorem~\ref{thm:exp-formula}\ref{thm:exp-formula:all},
\begin{align*}
1+A(x)&=e^{C(x)}=
\\
&=e^{\frac12\cdot x^2}=
\\
&=1+\tfrac12\cdot x^2+\tfrac1{2\cdot 2}\cdot x^4+\tfrac1{6\cdot 4}\cdot x^6+\dots+\tfrac1{n!\cdot 2^n}\cdot x^{2\cdot n}+\dots
\end{align*}
That is,
\[
\tfrac1{(2\cdot n-1)!}\cdot a_{2\cdot n-1}=0
\quad
\text{and}
\quad
\tfrac1{(2\cdot n)!}\cdot a_{2\cdot n}=\tfrac1{n!\cdot 2^n}\]
for any positive integer $n$.
In particular, \begin{align*}
a_{2\cdot n}&=\frac{(2\cdot n)!}{n!\cdot 2^n}=
\\
&=\frac{1\cdot 2\cdots (2\cdot n)}{2\cdot4 \cdots (2\cdot n)}=
\\
&=1\cdot 3\cdots (2\cdot n-1)=
\\
&=(2\cdot n-1)!!
\end{align*}

That is, $a_n=0$ for odd $n$ and $a_n=(n-1)!!$ for even $n$.
\qeds

\parbf{Remark.}
Note that by Corollary~\ref{cor:exp-formula}, we also have
\[A'(x)=[1+A(x)]\cdot x,\]
which is equivalent to the recursive identity
\[a_{n+1}=n \cdot a_{n-1}\]
in Exercise~\ref{ex:matching}\ref{ex:matching:recursion}.

\parbf{All matchings.}
Now let $\mathcal{S}$ is the set of two graphs $K_1$ or $K_2$.
In this case $c_1(\mathcal{S})\z=c_2(\mathcal{S})=1$, since $K_1$ and $K_2$ are spanning subgraph of itself.
Further we have that $c_n(\mathcal{S})=0$ for all $n\ge 3$ since $K_n$ contains no spanning subgraphs isomorphic to $K_1$ or $K_2$.

Therefore the exponential generating function of the sequence $c_n(\mathcal{S})$ is a polynomial of degree 2
\[C(x)=x+\tfrac12\cdot x^2.\]

Note that a matching in a graph $G$ can be identified with a spanning subgraph with all connected components isomorphic to  $K_1$ or $K_2$ --- that is, few isolated edges and few isolated vertexes.
If we denote by $a_n$ the number of all matchings and by $A(x)$ the corresponding exponential generating function then by  Theorem~\ref{thm:exp-formula}\ref{thm:exp-formula:all}, we get that
\[A(x)=e^{x+\frac12\cdot x^2}.\]
Applying Corollary~\ref{cor:exp-formula}, we also have
\[A'(x)=[1+A(x)]\cdot (1+x).\]
The latter is equivalent to the following recursive formula for $a_n$:
\[a_{n+1}=a_n+n\cdot a_{n-1}.\eqlbl{an+nan-1}\]
Since $a_1=1$ and $a_2=2$, we can easily find first few terms of this sequence:
\[1,2,4,10,26,\dots\]

\begin{thm}{Exercise}
Prove formula \ref{an+nan-1} directly --- without using generating functions.
Compare to Exercise~\ref{ex:deletion-deletion-total}\ref{ex:deletion-deletion-K}.
\end{thm}

\parbf{2-factors.}
Let $\mathcal{S}$ be the set of all cycles.

Note that $2$-factor of graph can be defined as a spanning subgraph with components isomorphic to cycles.
Denote by $a_n$ and $c_n$ the number of $2$-factors and spanning cycles in $K_n$.
Let $A(x)$ and $C(x)$ be the corresponding exponential generating functions.

\begin{thm}{Exercise}
\begin{enumerate}[(a)]
\item\label{ex:2-factor:cn} Show that $c_1=c_2=0$ and 
\[c_n=(n-1)!/2\]
for $n\ge 3$.
In particular 
\[c_1=0, c_2=0, c_3=1, c_4=3, c_5=12, c_6=60.\]
\item\label{ex:2-factor:recursive} Use the identity
\[A'(x)=[1+A(x)]\cdot C'(x)\]
to find $a_1,\dots, a_6$ using the part \ref{ex:2-factor:cn}.
\item Count the number of 2-factors in $K_1,\dots ,K_6$ and compare with the result in the part \ref{ex:2-factor:recursive}.
\item\label{ex:2-factor:C} Use part \ref{ex:2-factor:cn} to conclude that 
\[C(x)=\tfrac12\cdot\ln(1-x)-\tfrac12\cdot x-\tfrac14\cdot x^2\]
\item Use part \ref{ex:2-factor:C} and Theorem~\ref{thm:exp-formula}\ref{thm:exp-formula:all}
to show that
\[A(x)=\sqrt{1-x}/e^{\frac x2+\frac{x^2}4}.\]


\end{enumerate}

\end{thm}



\section*{Counting spanning forests}

Recall that a forest is a graph without cycles.
Assume we want to count the number of spanning forests in $K_n$;
denote by $a_n$ its number and by $c_n$ the number of connected spanning forests, that is, the number of spanning trees in $K_n$.

By Corollary~\ref{cor:exp-formula}, the following identity
\[A'(x)=[1+A(x)]\cdot C'(x)\]
holds for the corresponding exponential generating functions.

According to Cayley theorem, $c_n=n^{n-2}$;
therefore 
\[c_1=1, c_2=1, c_3=3, c_4=16,\dots\]
Applying the product formula \ref{eq:multiplication}, we can use $c_n$ to calculate $a_n$ recurrently:
\begin{align*}
a_1&=c_1=1,
\\
a_2&=c_2+a_1\cdot c_1=
\\&=1+1\cdot 1=2,
\\
a_3&=c_3+2\cdot a_1\cdot c_2+ a_2\cdot c_1=
\\
&=3+2\cdot1\cdot 1+2\cdot 1=7,
\\
a_4&=c_4+3\cdot a_1\cdot c_3+3\cdot a_2\cdot c_2+a_3\cdot c_1=
\\&=16+3\cdot 1\cdot 3+3\cdot 2\cdot 1+7\cdot 1=38
\\
&\dots
\end{align*}

It is instructive to check by hands there are exactly $38$ spanning forests in $K_4$.

For the general term of $a_n$ no simple formula is known,
however the recursive formula above provides sufficiently fast way to calculate its terms.


\section*{Counting connected subgraphs}

Let $a_n$ be the number of all subgraphs of $K_n$ and $c_n$ is the number of connected subgraphs of $K_n$
Assume $A(x)$ and $C(x)$ are the corresponding exponential generating functions.
These two series diverge for all $x\ne 0$;
nevertheless, the formula for formal power series in Theorem~\ref{thm:exp-formula}\ref{thm:exp-formula:all}
still holds and by Corollary~\ref{cor:exp-formula} we can write again
\[A'(x)=[1+A(x)]\cdot C'(x).\]

Note that $a_n=2^{\binom n2}$;
indeed, to describe a subgraph of $K_n$ we can choose any subset of $\tbinom n2$ edges of $K_n$, and $a_n$ is the total number of $\binom n2$ these independent choices.
In particular the first few terms of $a_n$ are
\[a_1=1,\quad a_2=2,\quad a_3=8,\quad a_4=64,\quad\dots\]

Applying the product formula \ref{eq:multiplication}, we can calculate the first few terms of~$c_n$:
\begin{align*}
c_1&=a_1=1
\\
c_2&=a_2-a_1\cdot c_1=
\\
&=2-1\cdot 1=1,
\\
c_3&=a_3-2\cdot a_1\cdot c_2- a_2\cdot c_1=
\\
&=8-2\cdot1\cdot 1-2\cdot 1=4,
\\
c_4&=a_4-3\cdot a_1\cdot c_3-3\cdot a_2\cdot c_2-1\cdot a_3\cdot c_1=
\\
&=64-3\cdot 1\cdot 4-3\cdot 2\cdot 1-1\cdot 8\cdot 1=38,
\\
&\dots
\end{align*}

Note that in the previous section we found $a_n$ from $c_n$ and now we go in the opposite direction.
For the general term of $c_n$ no closed formula is known,
but the recursive formula is nearly as good.

\section*{Remarks}

Let us mention another application of exponential generating functions.

Assume $r_n$ denotes the number of rooted spanning trees in $K_n$.
(A tree with one marked vertex is called \index{rooted tree}\emph{rooted tree} and the marked vertex is called its {}\emph{root}).
Then it is not hard to see that the exponential generating function of $r_n$ satisfies the following identity
\[R(x)=x\cdot e^{R(x)}.
\eqlbl{eq:rooted}\]
By Lagrange inversion theorem, formula \ref{eq:rooted} implies that $r_n=n^{n-1}$.

Since in any spanning tree of $K_n$ we have $n$ choices for the root, we have that
\[r_n=n\cdot s(K_n).\]
This way we get another proof of the Cayley formula (\ref{thm:cayley}) \[s(K_n)=n^{n-2}.\]

The method of generating function was introduced and widely used by Leonard Euler;
the term {}\emph{generating function} was coined latter by Pierre Laplace.
For more on the subject we recommend a classical book of Frank Harary and Edgar Palmer \cite{harary-palmer}.
